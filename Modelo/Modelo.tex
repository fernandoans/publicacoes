\documentclass[a4paper,11pt]{article}

% Identificação
\newcommand{\pbtitulo}{Modelo Doc}
\newcommand{\pbversao}{1.0}
\usepackage{../sty/tutorial}
\usepackage{lipsum}

%----------------------------------------------------------------------
% Início do Documento
%-----------------------------------------------------------------------
\begin{document}

Caixa de Listagem:
\begin{lstlisting}[]
Codigo de espacamento (nao use acentos aqui):

\vspace{-3em}
\\[3mm]
\end{lstlisting}

\begin{lstlisting}[]
\end{lstlisting}

% TECLAS
\keystroke{$x \lessgtr y$} The quick brown fox jumps over the lazy dog.
Colocações da Calculadora
\keystroke{Page $\uparrow$} \keystroke{Esc} \keystroke{F1}
Colocações da Calculadora
\keystroke{CHS} \keystroke{\ 1 } \keystroke{\ 2 } \keystroke{Enter}
\keystroke{$\bigtriangleup$\%} \keystroke{$y^x$} \keystroke{$\sqrt{x}$}
\keystroke{$\sum$} \keystroke{$\hat{X}$} \keystroke{$\bar{X}$}

	
\maketitle % mostrar o título
\thispagestyle{fancy} % habilitar o cabeçalho/rodapé das páginas

\begin{abstract}
	% O primeiro caractere deve vir com \initial{}
	\initial{M}\textbf{odelo de documento. \lipsum[4-1]}
\end{abstract}

Valores:
Circumflexo -> $\string^$

\section{Exemplos}

Fórmulas: \\
Interseção ou União: $A \cap B \cup C$ \\
Fração mais atraente: $\nicefrac{X}{Y}$ 

Isso aqui é uma nota\footnote{Na verdade essa é a nota} então NÃO USE PARENTESES.

Sendo: \vspace{-1em}
\begin{itemize}[nolistsep]
	\item $l_{inf}$ limite inferior.
	\item $n$ somatório das frequências simples.
	\item $f_{ac.Ant}$ frequência acumulada até a da classe anterior a da mediana.
	\item $f_{md}$ frequência simples.
	\item $h$ amplitude do intervalo.
\end{itemize}

Separação ideal para título de um MENU: \\
A partir do menu: Opção 1 $\triangleright$ Opção 2 $\triangleright$ Opção 3.

Comando qualquer:
{\ttfamily\$ sudo apt install docker docker.io} \\
{\ttfamily set PATH=\%PATH\%;C:$\setminus$Users$\setminus$xpto$\setminus$Anaconda3}

{\ttfamily MongoCursor<Document> cursor = getCol().find().iterator(); \\
	while (cursor.hasNext()) \{ \\
	\phantom{x}\hspace{4pt} System.out.println(cursor.next().toJson()); \\
	\} \\
	cursor.close(); }

\begin{quotation}
	Exemplo de Frase (Autor)
\end{quotation}

LIVROS

\begin{note}[Socorro]{}
	\lipsum[4-1]
\end{note}

APOSTILA

\begin{theo}[]{}
	\lipsum[4-1]
\end{theo}

\begin{theo}[Pode ser também com título]{}
	\lipsum[4-1]
\end{theo}

Minhas variáveis:

\opcmenu{Opção1 } $\triangleright$ \opcmenu{Opção2 } $\triangleright$ \opcmenu{Final}.

\aspas{ } útil para expaço em branco nos códigos, tipo isso

\codigo{funcao(\aspas{ })}

Clicar no botão \opcbotao{OK}.

Para figuras contínuas ao texto usar [H]. Exemplo de figura:
\begin{figure}[H]
	\centering
	\includegraphics[width=0.15\textwidth]{imagens/exemplo.jpg}
	\caption{Legenda}
\end{figure}

Exemplo de figura na mesma linha:

\begin{figure}[ht]
	\begin{minipage}[b]{0.45\linewidth}
		\centering
		\includegraphics[width=\textwidth]{filename1}
		\caption{default}
		\label{fig:figure1}
	\end{minipage}
	\hspace{0.5cm}
	\begin{minipage}[b]{0.45\linewidth}
		\centering
		\includegraphics[width=\textwidth]{filename2}
		\caption{default}
		\label{fig:figure2}
	\end{minipage}
\end{figure}

\begin{minipage}{\textwidth}
	\vspace{5pt}
	\begin{wrapfigure}{l}{0.15\textwidth}
		\vspace{-\baselineskip}
		\includegraphics[width=0.7\linewidth]{imagens/exemplo.jpg} 
	\end{wrapfigure}
	Aqui pode ir qualquer texto porém esse texto deve ser suficientemente grande para ocupar toda a área que a figura ocupar. Ou então use espaçadores, as vezes pode até dar certo, outras vezes precisamos reduzir o tamanho da figura.
\end{minipage}

\begin{minipage}{\textwidth}
	\vspace{5pt}
	\begin{wrapfigure}{t}{0.15\textwidth}
		\vspace{-\baselineskip}
		\includegraphics[width=0.7\linewidth]{imagens/exemplo.jpg} 
	\end{wrapfigure}
	Aqui pode ir qualquer texto porém esse texto deve ser suficientemente grande para ocupar toda a área que a figura ocupar. Ou então use espaçadores, as vezes pode até dar certo, outras vezes precisamos reduzir o tamanho da figura. \\[3mm]
\end{minipage}

Exemplo de Tabela:
\begin{table}[H]
	\centering 
	\begin{tabular}{c | L{3cm} | C{3cm} | R{3cm} }
		\textbf{Prefixo} & \textbf{Valor} & \textbf{Forma Padrão} & \textbf{Símbolo} \\
		\hline
		milli & 0,001 & $10^{-3}$ & m \\
		micro & 0,000001 & $10^{-6}$ & $\mu$ \\
		Ângström & 0,0000000001 & $10^{-10}$ & Â \\
		pico & 0,000000000001 & $10^{-12}$ & p \\
	\end{tabular}
\end{table}

\begin{center}
	\begin{tabular}{ c|c|c } 
		\hline
		cell1 & cell2 & cell3 \\ 
		cell4 & cell5 & cell6 \\ 
		cell7 & cell8 & cell9 \\ 
		\hline
	\end{tabular}
\end{center}

Espaçamento antes de lista: \vspace{-1em}
\begin{itemize}
	\item \textbf{Teste0}: Teste0
    \item \textbf{Teste1}: Teste1
    \item \textbf{Teste2}: Teste2
\end{itemize}

Sem espaçamento na lista:

\begin{enumerate}[nolistsep]
	\item \textbf{Teste0}: Teste0
	\item \textbf{Teste1}: Teste1
	\item \textbf{Teste2}: Teste2
\end{enumerate}

% se quiser colocar o espaçameto geral use o comando
% \setlist[itemize]{noitemsep}
% no config.sty

\section{Conclusão}

Sou um entusiasta do mundo \textbf{Open Source} e novas tecnologias. Qual a diferença entre Livre e Open Source? \underline{Livre} significa que esta apostila é gratuita e pode ser compartilhada a vontade. \underline{Open Source} além de livre todos os arquivos que permitem a geração desta (chamados de arquivos fontes) devem ser disponibilizados para que qualquer pessoa possa modificar ao seu prazer, gerar novas, complementar ou fazer o que quiser. Os fontes da apostila (que foi produzida com o LaTex) está disponibilizado no GitHub \cite{github}. Veja ainda outros artigos que publico sobre tecnologia através do meu Blog Oficial \cite{fernandoanselmo}.

%--------------------------------------------------------------------------
% REFERÊNCIAS
%--------------------------------------------------------------------------
\begin{thebibliography}{3}
		\bibitem{fernandoanselmo} 
	Fernando Anselmo - Blog Oficial de Tecnologia \\
	\url{http://www.fernandoanselmo.blogspot.com.br/}
	
	\bibitem{publicacao} 
	Encontre essa e outras publicações em \\
	\url{https://cetrex.academia.edu/FernandoAnselmo}
	
	\bibitem{github} 
	Repositório para os fontes da apostila \\
	\url{https://github.com/fernandoans/publicacoes}
\end{thebibliography}


\end{document}