\documentclass[a4paper,11pt]{article}

% Identificação
\newcommand{\pbtitulo}{Palitos Matemáticos}
\newcommand{\pbversao}{1.1}
\usepackage{criarpalitos}

\begin{document}
  \maketitle % mostrar o título
  \thispagestyle{fancy} % habilitar o cabeçalho/rodapé das páginas
  
  \section*{Fórmulas Matemáticas com Fósforos}
  Estimule sua criatividade e habilidades matemáticas à medida que você mergulha nesse intrigante mundo de quebra-cabeças com palitos de fósforo. Pode ser necessário reposicionar um único palito, ou em outras situações, mais de um. Utilize seu raciocínio lógico para solucionar as equações matemáticas de acordo com as instruções fornecidas em cada problema.
  
  \subsection*{Problema 1}
  A equação é corrigida ao mover 1 palito:
  
  \digitThree \plus \digitTwo \equal \digitSeven

  \subsection*{Problema 2}
  A equação é corrigida ao mover 1 palito:

  \digitEight \minus \digitFive \equal \digitZero

  \subsection*{Problema 3}
  A equação é corrigida ao mover 1 palito:

  \digitSeven \minus \digitThree \equal \digitTwo

  \subsection*{Problema 4}
  A equação é corrigida ao mover 1 palito:

  \digitSix \plus \digitFour \equal \digitFour

  \subsection*{Problema 5}
  A equação é corrigida ao mover 1 palito para que o resultado seja 0:

  \digitOne \plus \digitSeven \equal \digitZero

  \subsection*{Problema 6}
  A equação é corrigida ao mover 1 palito para que o resultado seja 0:

  \digitNine \minus \digitFive \minus \digitOne \equal \digitZero

  \subsection*{Problema 7}
  A equação é corrigida ao mover 2 palitos:

  \digitOne \minus \digitOne \equal \digitNine

  \subsection*{Problema 8}
  A equação é corrigida ao mover 1 palito:

  \digitNine \plus \digitThree \equal \digitNine

  \subsection*{Problema 9}
  A equação é corrigida ao mover 1 palito:

  \digitEight \minus \digitEight \equal \digitNine

  \subsection*{Problema 10}
  A equação é corrigida ao mover 2 palitos:

  \digitFour \digitOne \minus \digitTwo \equal \digitFour
  
  \subsection*{Problema 11}
  A equação é corrigida ao mover 2 palitos:

  \digitZero \plus \digitZero \equal \digitOne \digitTwo
  
  \subsection*{Problema 12}
  A equação é corrigida ao mover 1 palito:

  \digitEight \minus \digitFour \equal \digitZero
  
  \subsection*{Problema 13}
  A equação é corrigida ao mover 1 palito:

  \digitTwo \minus \digitThree \equal \digitSix
  
  \subsection*{Problema 14}
  A equação é corrigida ao mover 1 palito:

  \digitThree \minus \digitTwo \equal \digitSeven
  
  \subsection*{Problema 15}
  A equação é corrigida ao mover 1 palito:

  \digitZero \minus \digitZero \equal \digitSix

  \subsection*{Problema 16}
  A equação é corrigida ao mover 1 palito:

  \digitThree \plus \digitOne \equal \digitEight

  \subsection*{Problema 17}
  A equação é corrigida ao mover 2 palitos:

  \digitSix \minus \digitZero \equal \digitEight \digitOne
  
  \subsection*{Problema 18}
  A equação é corrigida ao mover 1 palito para que o resultado seja -1:

  \digitEight \minus \digitOne \equal \minus \digitOne   
     
  \subsection*{Problema 19}
  A equação é corrigida ao mover 1 palito:

  \digitSix \plus \digitFour \equal \digitFour

  \subsection*{Problema 20}
  Transforme para o maior número possível ao mover 1 palito:

  \digitFive \digitZero \digitEight

  \subsection*{Problema 21}
  A equação é corrigida com adição de 2 palitos:

  \digitOne \plus \digitOne \digitOne \equal \digitTwo \digitFour

  \subsection*{Problema 22}
  A equação é corrigida ao mover 1 palito:

  \digitTwo \plus \digitThree \equal \digitSix
  
  \subsection*{Problema 23}
  A equação é corrigida ao mover 2 palitos para que o resultado seja 12:

  \digitZero \plus \digitZero \equal \digitOne \digitTwo

  \subsection*{Problema 24}
  A equação é corrigida ao mover 2 palitos:

  \digitFive \plus \digitThree \equal \digitFive

  \subsection*{Problema 25}
  A equação é corrigida ao mover 1 palito:

  \digitOne \plus \digitSeven \equal \digitZero

  \subsection*{Problema 26}
  A equação é corrigida ao mover 3 palitos:

  \digitTwo \minus \digitTwo \equal \digitFive
  
  \subsection*{Problema 27}
  A equação é corrigida ao mover 1 palito:

  \digitZero \plus \digitThree \equal \digitTwo
  
  \subsection*{Problema 28}
  A equação é corrigida ao mover 1 palito:

  \digitZero \plus \digitOne \plus \digitTwo \equal \digitSeven

%--------------------------------------------------------------------------
% REFERÊNCIAS
%--------------------------------------------------------------------------
\begin{thebibliography}{4}
		\bibitem{fernandoanselmo} 
	Fernando Anselmo - Blog Oficial de Tecnologia \\
	\url{http://www.fernandoanselmo.blogspot.com.br/}
	
	\bibitem{publicacao} 
	Encontre essa e outras publicações em \\
	\url{https://cetrex.academia.edu/FernandoAnselmo}
	
	\bibitem{github} 
	Repositório para os fontes da apostila \\
	\url{https://github.com/fernandoans/publicacoes}
\end{thebibliography}
    
\end{document}