\section*{Operações Matemáticas}	
Essas são as \textbf{Funções Aritméticas}: \vspace{-1em}
\begin{description}
	\item[Somar:] Para resolver a expressão $4 + 3$, seguir a seguinte sequencia: \keystroke{$4$} \keystroke{$ENTER$} \keystroke{$3$} \keystroke{$+$}, e como resultado teremos no visor o valor 7.
	\item[Subtrair:] Para resolver a expressão $5 - 3$, seguir a seguinte sequencia: \keystroke{$5$} \keystroke{$ENTER$} \keystroke{$3$} \keystroke{$-$}, e como resultado teremos no visor o valor 2.
	\item[Multiplicar:] Para resolver a expressão $7 \times 3$, seguir a seguinte sequencia: \keystroke{$7$} \keystroke{$ENTER$} \keystroke{$3$} \keystroke{$\times$}, e como resultado teremos no visor o valor 21.
	\item[Dividir:] Para resolver a expressão $10 \div 2$, seguir a seguinte sequencia: \keystroke{$10$} \keystroke{$ENTER$} \keystroke{$2$} \keystroke{$\div$}, e como resultado teremos no visor o valor 5.
\end{description}

Essas são as \textbf{Funções Algébricas}: \vspace{-1em}
\begin{itemize}
	\item número \keystroke{$g$} \keystroke{$FRAC$} - isolar a parte fracionária
	\item número \keystroke{$g$} \keystroke{$INTG$} - isolar a parte inteira
	\item número \keystroke{$1/x$} - inverso
	\item número \keystroke{$g$} \keystroke{$n!$} - fatorial
	\item número \keystroke{$g$} \keystroke{$\sqrt{x}$} - raiz quadrada
	\item número \keystroke{$Enter$} expoente \keystroke{$y^x$} - potenciação
	\item número \keystroke{$Enter$} base \keystroke{$1/x$} \keystroke{$y^x$} - raiz qualquer
\end{itemize}

Essas são as \textbf{Funções Logarítmicas}: \vspace{-1em}
\begin{itemize}
	\item número \keystroke{$g$} \keystroke{$LN$} - logaritmo natural
	\item número \keystroke{$g$} \keystroke{$e^x$} - antilogaritmo (É a função inversa do logaritmo)
	\item número \keystroke{$g$} \keystroke{$LN$} base \keystroke{$g$} \keystroke{$LN$} \keystroke{$\div$} - logaritmo em qualquer base
	\item resultado \keystroke{$Enter$} base \keystroke{$x \lessgtr y$} \keystroke{$y^x$} - antilogaritmo em qualquer base
\end{itemize}

\subsection*{Percentual}
Essas são as operações básicas para se trabalhar com percentual: \vspace{-1em}
\begin{itemize}
	\item número \keystroke{$Enter$} número \keystroke{$\%$} - Calculo Básico = [baseP]
	\item valP \keystroke{$-$} - Subtrai o percentual do total
	\item valP \keystroke{$+$} - Aumenta o percentual do total
	\item número \keystroke{$Enter$} número \keystroke{$\bigtriangleup \%$} - Diferença Percentual (somar com 100 para obter o valor percentual)
	\item número \keystroke{$Enter$} valP \keystroke{$\%T$} - Percentagem do Total (númT = Número Total valP = Valor Parcial)
\end{itemize}

\textbf{Problema 1}: Um imóvel foi comprado por R\$ 110.000,00 e vendido por R\$ 138.400,00. Qual foi o percentual de lucro? (para agilizar a entrada de valores podemos dividi-los por 1.000)

\keystroke{$1$} \keystroke{$1$} \keystroke{$0$} \keystroke{$Enter$} \keystroke{$1$} \keystroke{$3$} \keystroke{$8$} \keystroke{$.$} \keystroke{$4$} \keystroke{$\bigtriangleup \%$} 

O ganho foi de 25,82\%

\textbf{Problema 2}: Um título de capitalização possui seu valor aumentado em 0,5\% após 1 ano, considerando que foram comprados 10 títulos no valor de R\$ 50,00 cada. Qual será o valor resgatado após o período estabelecido?

\keystroke{$5$} \keystroke{$0$} \keystroke{$Enter$} \keystroke{$0$} \keystroke{$.$} \keystroke{$5$} \keystroke{$\%$} \keystroke{$+$} \keystroke{$1$} \keystroke{$0$} \keystroke{$\times$}

Multiplicamos por 10 ao final pois foram comprados 10 títulos, o valor resgatado será de R\$ 502,50, ou seja, R\$ 2,50 a mais.

\textbf{Problema 3}: Dois amigos montaram uma Empresa, o primeiro entrou com R\$ 500,00 e o segundo com R\$ 300,00. Qual o percentual de participação dos sócios no lucro da Empresa?

1. Capital Total: \keystroke{$5$} \keystroke{$0$} \keystroke{$0$} \keystroke{$Enter$}  \keystroke{$3$} \keystroke{$0$} \keystroke{$0$} \keystroke{$+$}

2. Participação sócio 1: \keystroke{$5$} \keystroke{$0$} \keystroke{$0$} \keystroke{$\%T$}

3. Participação sócio 2: \keystroke{$CLX$} \keystroke{$3$} \keystroke{$0$} \keystroke{$0$} \keystroke{$\%T$}  

Sócio 1 com \textbf{62,50\%} e Sócio 2 com \textbf{37,50\%}.

\textbf{Problema 4}: Um eletrodoméstico que estava sendo vendido por R\$ 340,00 foi majorado\footnote{Acréscimo no preço do bem} em 8\%. Qual o novo preço de venda?

\keystroke{$3$} \keystroke{$4$} \keystroke{$0$} \keystroke{$Enter$} \keystroke{$8$} \keystroke{$\%$} \keystroke{$+$}

O novo preço de venda é \textbf{R\$ 367,20}.

\textbf{Problema 5}: Foi recebido um salário de R\$ 935,00 após um reajuste de 5\%. Qual era o valor do salário anterior?

\keystroke{$9$} \keystroke{$3$} \keystroke{$5$} \keystroke{$Enter$} \keystroke{$1$} \keystroke{$Enter$} \keystroke{$5$} \keystroke{$\%$} \keystroke{$+$} \keystroke{$\div$}

O salário anterior era de \textbf{R\$ 890,48}.

\textbf{Problema 6}: O faturamento mensal de uma empresa é de R\$ 800,00, o valor das vendas a vista, R\$ 481,00. Qual a porcentagem de participação das vendas a vista em relação ao total?

\keystroke{$8$} \keystroke{$0$} \keystroke{$0$} \keystroke{$Enter$} \keystroke{$4$} \keystroke{$8$} \keystroke{$1$} \keystroke{$\%T$} 

A porcentagem de participação é \textbf{60,13\%}.

\textbf{Problema 7}: Calcular a evolução o percentual de faturamento para uma empresa conforme a seguinte tabela:
\begin{table}[H]
	\centering 
	\begin{tabular}{L{2cm} | R{3cm} }
		\textbf{Mês} & \textbf{Valor (Em mil R\$)} \\
		\hline
		Janeiro & 58 \\
		Fevereiro & 66 \\
		Março & 72 \\
		Abril & 67 \\
	\end{tabular}
\end{table}

1. De Janeiro a Fevereiro: \\
\keystroke{$5$} \keystroke{$8$} \keystroke{$Enter$} \keystroke{$6$} \keystroke{$6$} \keystroke{$\bigtriangleup \%$}

2. De Fevereiro a Março: \\
\keystroke{$6$} \keystroke{$6$} \keystroke{$Enter$} \keystroke{$7$} \keystroke{$2$} \keystroke{$\bigtriangleup \%$}

3. De Março a Abril: \\
\keystroke{$7$} \keystroke{$2$} \keystroke{$Enter$} \keystroke{$6$} \keystroke{$7$} \keystroke{$\bigtriangleup \%$}

E teremos os seguinte percentuais: \textbf{13,79\%}, \textbf{9,09\%} e \textbf{-6,94\%}.

\subsection*{Números com mais de 10 dígitos}
O visor da HP-12C comporta até 10 dígitos. Para introduzir um número com mais de dez dígitos (por exemplo 500.000.000.000), procedemos da seguinte maneira: \vspace{-1em}
\begin{enumerate}
	\item Anote esse número em notação científica (5e11)
	\item Teclar a mantissa: \keystroke{$5$}
	\item Pressionar a tecla \keystroke{$RND$}
	\item Teclar o expoente: \keystroke{$11$}
\end{enumerate}
	
Outra forma é utilizar as teclas \keystroke{$f$} \keystroke{$.$} para expressar as potencias de 10. Por exemplo o numero 4.069.948.757. Pressionar na sequencia: \keystroke{$4$} \keystroke{$0$} \keystroke{$6$} \keystroke{$9$} \keystroke{$9$} \keystroke{$4$} \keystroke{$8$} \keystroke{$7$} \keystroke{$5$} \keystroke{$7$} \keystroke{$f$} \keystroke{$.$} e no visor aparece: \textbf{4,069948 09}