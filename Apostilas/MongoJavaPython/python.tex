\section{Python}
Python é uma linguagem de programação de alto nível, interpretada a partir de um script, Orientada a Objetos e de tipagem dinâmica. Foi lançada por Guido van Rossum em 1991. Não pretendo nesta apostila COMPARAR essa linguagem com Java (espero que nunca o faça), fica claro que os comandos são bem mais fáceis porém essas linguagens possuem diferentes propósitos.

Todos os comandos descritos abaixo foi utilizado no JupyterLab \cite{jupyteroficial}, então basta abrir um Notebook e digitá-los em cada célula conforme se apresentam.

\subsection{Proceder a Conexão}
Baixar o pacote necessário: \\
\codigo{ !pip install pymongo}

Importar os pacotes necessários: \\
\codigo{ from pymongo import MongoClient \\
	import random}

Neste caso estamos utilizando o pacote \textbf{random} somente para criarmos o mesmo exemplo já visto e escolher uma nota aleatória para casa aluno.

Nos conectamos ao servidor desta forma: \\
\codigo{ cliente = MongoClient('localhost', 27017)}
Ou: \\
\codigo{ cliente = MongoClient('mongodb://localhost:27017/')}

Listar as bases disponíveis: \\
\codigo{ cliente.list\_database\_names()) }

Nos conectamos a uma base desta forma: \\
\codigo{ db = cliente.escola}
Ou: \\
\codigo{ db = cliente['escola']}

Listar as coleções disponíveis: \\
\codigo{ cliente.list\_collection\_names()) }

Nos conectamos a uma coleção desta forma: \\
\codigo{ col = db.aluno}
Ou: \\
\codigo{ col = db['aluno']}

\subsection{Inserir documentos}
Inserir um único documento é uma questão de criar um dicionário e enviá-lo para a coleção: \\
\codigo{ mario = \{ "nome": "Mario da Silva", "nota": random.randint(1,11) \} \\
	col.insert\_one(mario) }

Inserir vários documentos é necessário criar uma lista de dicionários e enviar a lista para a coleção: \\
\codigo{ alunos = [ \\
	\phantom{x}\hspace{4pt} \{ "nome": "Aline Moraes", "nota": random.randint(1,11) \}, \\
	\phantom{x}\hspace{4pt} \{ "nome": "Soraya Gomes", "nota": random.randint(1,11) \} \\
	] \\
	col.insert\_many(alunos)
}

\subsection{Encontrar documentos}
Listar toda a coleção: \\
\codigo{ for doc in col.find(\{\}): \\
	\phantom{x}\hspace{4pt} print(doc)}

Listar toda a coleção de modo ordenado ascendente (ou descendente - valor -1): \\
\codigo{for doc in col.find(\{\}).sort("campo",1): \\
	\phantom{x}\hspace{4pt} print(doc)}

Quantos documentos existem na coleção: \\
\codigo{ col.count\_documents(\{\})}

Trazer o primeiro documento: \\
\codigo{ col.find\_one()}

Trazer um determinado documento: \\
\codigo{ col.find\_one(\{"nome": "Aline Moraes"\})}

Limitar a quantidade de documentos buscados (no caso 5): \\
\codigo{for doc in col.find(\{\}).limit(5): \\
	\phantom{x}\hspace{4pt} print(doc)}

Mostrar um determinado campo (e somente ele): \\
\codigo{for doc in col.find(\{\}): \\
	\phantom{x}\hspace{4pt} print(doc['col'])}

Trazer os documentos que possuem a nota maior que 5 e menor que 7: \\
\codigo{ for doc in col.find(\{"nota": \{"\$gt": 5, "\$lt": 7\}\}): \\
	\phantom{x}\hspace{4pt} print(doc)}

\subsection{Atualizar documentos}
* O lado da esquerda é o filtro de consulta e o lado do SET são os campos a alterar.

Alterar um documento que possui o nome "Mario da Silva": \\
\codigo{ col.update\_one(\{"nome": "Mario da Silva"\}, \{"\$set": \{"nota": 8\}\})}

Alterar os documentos que possuem a nota menor que 5: \\
\codigo{ col.update\_many(\{'nota': \{'\$lt': 5\}\}, \{'\$set': \{'nota': 4\}\})}

Eliminar um documento que possui o nome "Mario da Silva": \\
\codigo{ col.delete\_one(\{"nome": "Mario da Silva"\})}

Eliminar os documentos que possuem a nota menor que 5: \\
\codigo{ col.delete\_many(\{'nota': \{'\$lt': 5\}\})}

\subsection{Encerrar}
É boa prática fechar a base de dados: \\
\codigo{ cliente.close()}

Mas antes de encerramos realmente vejamos o seguinte programa completo em linguagem Python:

\begin{lstlisting}[]
from pymongo import MongoClient
from random import randint

# Passo 1: Conectar ao Mongo
cliente = MongoClient(port=27017)
db = cliente.negocio

# Passo 2: Criar Amostras de Dados
nomes = ['Kitchen', 'Espiritual', 'Mongo', 'Tastey', 'Big', 'Jr', 'Filho', 'City', 'Linux'
         'Tubarão', 'Gado', 'Sagrado', 'Solo', 'Sumo', 'Lazy', 'Fun', 'Prazer', 'Gula']
tipo_emp = ['LLC', 'Inc', 'Cia', 'Corp.']
tipo_coz = ['Pizza', 'Bar', 'Fast Food', 'Italiana', 'Mexicana',
            'Americana', 'Sushi', 'Vegetariana', 'Churrascaria']

for x in range(1, 501):
  nome1 = nomes[randint(0, (len(nomes)-1))]
  nome2 = nomes[randint(0, (len(nomes)-1))]
  tipoE = tipo_emp[randint(0, (len(tipo_emp)-1))]
  negocio = {
	'nome': nome1 + ' ' + nome2 + ' ' + tipoE,
	'nota': randint(1, 5),
	'cozinha': tipo_coz[randint(0, (len(tipo_coz)-1))]
  }  

# Passo 3: Inserir o objeto negócio no banco
result = db.restaurante.insert_one(negocio)

# Passo 4: Mostrar no console o Object ID do Documento
print('Criado {0} de 500 como {1}'.format(x, result.inserted_id))

# Passo 5: Mostrar mensagem final
print('500 Novos Negócios Culinários foram criados...')
cliente.close()
\end{lstlisting}

O programa está auto-documentado e criar uma base com 500 registros.