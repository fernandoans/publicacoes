\documentclass[a4paper,11pt]{article}

% Identificação
\newcommand{\pbtitulo}{Cassandra com Java e Python}
\newcommand{\pbversao}{1.1}

\usepackage{../sty/tutorial}

%----------------------------------------------------------------------
% Início do Documento
%----------------------------------------------------------------------
\begin{document}
	
\maketitle % mostrar o título
\thispagestyle{fancy} % habilitar o cabeçalho/rodapé das páginas

%----------------------------------------------------------------------
% RESUMO DO ARTIGO
%----------------------------------------------------------------------
	
\begin{abstract}
	\initial{A}\textbf{pache Cassandra é um Sistema para Banco de Dados distribuído e altamente escalável de segunda geração, que reúne a arquitetura do DynamoDB, da Amazon Web Services e modelo de dados baseado no BigTable, do Google. O Modelo de Dados do Cassandra é um amplo armazenamento de colunas e essencialmente um híbrido entre o que conhecemos por valor-chave e um sistema de gerenciamento tabular. Consiste com um armazenamento de linha particionado permitindo uma consistência ajustável. Linhas são organizadas em tabelas; o primeiro componente da chave primária de uma tabela é a chave de partição; dentro de uma partição, as linhas são agrupadas pelas colunas restantes da chave. Outras colunas podem ser indexadas separadamente da chave primária. Neste tutorial veremos o que vem a ser o banco Cassandra \cite{cassandraoficial} e como proceder sua utilização utilizando como pano de fundo a linguagem de programação Java \cite{javaoficial} e Python \cite{pythonoficial}.}
\end{abstract}

%-----------------------------------------------------------------------------
% CONTEÚDO DO ARTIGO
%-----------------------------------------------------------------------------

\section{Parte inicial}
\textbf{Avinash Lakshman}, um dos autores do DynamoDB da Amazon, e Prashant Malik \footnote{Os desenvolvedores nomearam o banco de dados em homenagem ao profeta mitológico \textbf{Trojan Cassandra}, com alusões clássicas à maldição de um oráculo.} desenvolveram inicialmente o Cassandra no Facebook para gerenciar o recurso de pesquisa da caixa de entrada do Facebook. O Facebook lançou o Cassandra como um projeto de código aberto no código do Google em julho de 2008.  Em março de 2009, tornou-se um projeto da Incubadora Apache. Em 17 de fevereiro de 2010, ele se formou em um projeto de nível superior.
\begin{figure}[H]
	\centering
	\includegraphics[width=0.2\textwidth]{imagens/logo}
	\caption{Logo do Apache Cassandra}
\end{figure}

A Arquitetura Cassandra consiste nos seguintes componentes:
\begin{itemize}[nolistsep]
	\item \textbf{Node (Nó)}. É o componente básico dos dados, uma máquina onde os dados são armazenados.
	\item \textbf{DataCenter}. Uma coleção de nós relacionados. Pode ser físico ou virtual.
	\item \textbf{Cluster}. Contém um ou mais \textit{DataCenter}, ele pode se estender por vários locais.
	\item \textbf{Commit Log}. Cada operação de gravação é primeiro armazenada no \textit{log} (registro) de confirmação. Normalmente é utilizado para recuperação de falhas. 
	\item \textbf{MemTable}. Depois que os dados são gravados no \textit{log} de confirmação, são armazenados em uma \textit{Memory Table} (tabela em memória), que permanece lá até atingir o limite.
	\item \textbf{SSTable}. \textit{Sorted-String Table} é um arquivo de disco que armazena dados da \textit{MemTable} assim que atinge o limite. As SSTables são armazenadas em disco sequencialmente e mantidas para cada tabela do banco de dados. 
\end{itemize}

\subsection{Estratégias de replicação de dados}
Uma das partes mais importantes para entedermos o Cassandra é sua capacidade de "Replicação dos Dados", isso não é opcional mas um recurso obrigatório para garantir que nenhum dado seja perdido devido a falha de hardware ou rede. Uma estratégia de replicação determina em quais nós colocar réplicas. Cassandra oferece duas estratégias de replicação diferentes.

\textbf{Simple Strategy - Estratégia Simples}
\begin{figure}[H]
	\centering
	\includegraphics[width=0.2\textwidth]{imagens/simpleStrategy.jpg}
	\caption{Estrutura da Estratégia Simples}
\end{figure}

Utilizada quando se possui um único \textit{DataCenter}. É colocada a primeira réplica no nó selecionado pelo particionador. Um particionador determina como os dados são distribuídos pelos nós do cluster (incluindo suas réplicas). Depois disso, as réplicas restantes são colocadas no sentido horário no anel do nó.

\textbf{Network Topology Strategy - Estratégia de topologia de rede}
\begin{figure}[H]
	\centering
	\includegraphics[width=0.4\textwidth]{imagens/networkStrategy.jpg}
	\caption{Estrutura da estratégia de topologia de rede}
\end{figure}

Utilizada quando se possui implantações em vários \textit{Datacenters}. Essa estratégia coloca réplicas no mesmo \textit{Datacenter}, percorrendo o anel no sentido horário até chegar ao primeiro nó em outro \textit{rack}. Isso ocorre porque às vezes podem ocorrer falhas ou problemas no \textit{rack}. Então, as réplicas em outros nós podem fornecer dados. Esta estratégia é altamente recomendada para fins de escalabilidade e expansão futura. 

\subsection{Criar o contêiner Docker}
A forma mais simples de termos o Apache Cassandra é através de um contêiner no Docker, assim facilmente podemos ter várias versões do banco instalada e controlar mais facilmente qual banco está ativo ou não. E ainda colhemos o benefício adicional de não termos absolutamente nada deixando sujeira em nosso sistema operacional ou áreas de memória.

Baixar a imagem oficial: \\
\codigo{\$ docker pull cassandra}

Criar uma instância do banco em um contêiner: \\
\codigo{\$ docker run -it -p 9042:9042 --name meu-cassandra -d \\ -v /home/[seuUsuario]/cassandra/data/node1:/var/lib/cassandra/data cassandra}

Nessa instância criada estamos associando a porta 9042 para acessarmos o banco e um volume em nossa máquina para armazenar os dados do contêiner.

Podemos acessar o Shell de comandos do Cassandra no contêiner (espere um pouco até o banco levantar): \\
\codigo{\$ docker exec -it meu-cassandra cqlsh}
\begin{lstlisting}[]
cqlsh> SELECT * FROM system_schema.keyspaces;
cqlsh> exit
\end{lstlisting}

Podemos parar o contêiner com: \\
\codigo{\$ docker stop meu-cassandra} 

Ou iniciá-lo novamente: \\
\codigo{\$ docker start meu-cassandra} 

\section{Redis-Cli}
Diferente dos outros bancos o Redis preza por ser extremamente simples, podemos selecionar qualquer base com o seguinte comando: \\
\codigo{> select N}

Onde este N é simplesmente o número da base, a primeira inicia com 0, sendo assim: \\
\codigo{> select 0}

Este comando verifica a quantidade de elementos: \\
\codigo{> dbsize}

Outro simples comando para saber que está tudo OK é: \\
\codigo{> ping}

E a resposta deve ser \textbf{PONG} o que significa "estamos sem problemas". Para inserirmos um simples elemento: \\
\codigo{> set valor 100}

E para resgatá-lo: \\
\codigo{> get valor}

Qualquer tipo para o Redis pode ser tratado como uma String, assim para obtermos o tamanho de uma determinada chave: \\
\codigo{> strlen valor}

Mas observamos também que por ser um número podemos realizar operações como incrementar o valor: \\
\codigo{> incr valor}

Ou decrementar o valor: \\
\codigo{> decr valor}

O mais curioso é que podemos ainda concatenar (anexar) mais informações: \\
\codigo{> decr 200}

E agora nossa chave tem o valor "100200". E podemos adicionar mais valores: \\
\codigo{> incrby valor 50}

Ou diminuir valores: \\
\codigo{> decrby valor 50}

\subsection{Chave-Valor}
O maior uso do Redis e o armazenamento de informações como uma tabela no formato chaves-valores: \\
\codigo{> hmset identificacao nome Fernando cargo Analista materia Estatística}

Adicionar mais uma propriedade: \\
\codigo{> hset identificacao cidade Brasília}

Recuperar uma determinada propriedade: \\
\codigo{> hmget identificacao cargo}

Obter todos os valores: \\
\codigo{> hvals identificacao}

Obter todos os nomes das chaves: \\
\codigo{> hkeys identificacao}

Ou mais completo como: \\
\codigo{> hgetall identificacao}

Obter o tamanho: \\
\codigo{> hlen identificacao}

Eliminar uma propriedade: \\
\codigo{> hdel identificacao materia}

\subsection{Listas de valores}
Para criar uma chave como valores de lista e adicionar elementos a ela (a direita ou no final): \\
\codigo{> rpush tipos persa ragdoll bengali}

Ou então (a esquerda ou no início): \\
\codigo{> lpush tipos siamês}

Recuperar todos os tipos (se ainda não percebeu de gatos): \\
\codigo{> lrange tipos 0 -1}

Ou um determinado tipo (por exemplo o terceiro elemento - a lista inicia do 0): \\
\codigo{> lrange tipos 2 2}

Verificar o tamanho da lista: \\
\codigo{> llen tipos}

Eliminar determinado elemento: \\
\codigo{> lpop tipos 2}

\subsection{Conjuntos de valores}
Para criarmos uma chave como valores de conjuntos e adicionar elementos a ela: \\
\codigo{> sadd paises Brasil EUA Portugal Espanha}

Recuperar a quantidade de países: \\
\codigo{> scard paises}

Recuperamos todos os países: \\
\codigo{> smembers paises}

Ou obter um país aleatoriamente: \\
\codigo{> srandmember paises 1}

Eliminar determinado elemento: \\
\codigo{> srem paises 2}

\subsection{Movimentos e Exclusão}
Vericar se uma determinada chave existe: \\
\codigo{> exists paises}

Mover uma determinada chave para outro banco: \\
\codigo{> move tipos 1}

Modificar o nome de uma determinada chave: \\
\codigo{> rename paises destinos}

Para excluirmos qualquer chave temos mais um comando extremamente simples: \\
\codigo{> del destinos}
\section{Linguagem Java}
Java é considerada a linguagem de programação orientada a objetos mais utilizada no Mundo, é a base para construção de ferramentas como Hadoop, Pentaho, Weka e muitas outras utilizados comercialmente. Foi desenvolvida na década de 90 por uma equipe de programadores chefiada por \textit{James Gosling} para o projeto Green, na empresa Sun Microsystems - tornou-se nessa época como a linguagem que os programadores mais baixaram e o sucesso foi instantâneo. Em 2008 o Java foi adquirido pela Oracle Corporation.

\subsection{Driver JDBC de Conexão}
Para proceder a conexão com Java, é necessário baixar um driver JDBC (Java Database Connection). Existem vários drivers construídos, porém o driver oficialmente suportado pelo Apache Cassandra se encontra no endereço: \url{https://docs.datastax.com/en/driver-matrix/doc/driver_matrix/common/driverMatrix.html}

Para utilizar o driver é necessário criar um projeto (Utilizaremos o \textbf{Spring Tool Suite 4}, porém pode utilizar qualquer outro editor de sua preferência).

No STS4 acessar a seguinte opção no menu: File $\triangleright$ New $\triangleright$ Java Project. Informar o nome do projeto (meucass), não esquecer de modificar a opção "Use an environment JRE" para a versão correta da Java Runtime desejada e pressionar o botão Finish. Se está tudo correto teremos a seguinte situação na aba \textit{Project Explorer}.

Vamos convertê-lo para um projeto Apache Maven. Clicar com o botão direito do mouse no projeto e acessar a opção: Configure $\triangleright$ Convert to Maven Project. Na janela apenas pressione o botão \textit{Finish}. Se tudo está correto observamos que o projeto ganhou uma letra \textbf{M} o que indica agora é um projeto padrão Maven. Então foi criado um arquivo chamado \textbf{pom.xml}. 

Acessar este arquivo e antes da tag BUILD, inserir a tag DEPENDENCIES:
\begin{lstlisting}[]
<dependencies>
 <dependency>
  <groupId>com.datastax.cassandra</groupId>
  <artifactId>cassandra-driver-core</artifactId>
  <version>3.1.0</version>
 </dependency>
 <dependency>
  <groupId>org.cassandraunit</groupId>
  <artifactId>cassandra-unit</artifactId>
  <version>3.0.0.1</version>
 </dependency>
 <dependency>
  <groupId>org.projectlombok</groupId>
  <artifactId>lombok</artifactId>
  <scope>provided</scope>
  <version>1.18.20</version>
 </dependency>
</dependencies>
\end{lstlisting}

Observamos que na pasta \textbf{Maven Dependencias} foi baixado a versão 3.1.0 do driver JDBC do Cassandra.

\subsection{Classe Livro}
Estamos prontos para testarmos a conexão entre porém vamos criar uma classe chamada \textbf{Livro} no pacote \textbf{meucass} e inserir nesta a seguinte codificação:
\begin{lstlisting}[]
package meucass;

import java.util.UUID;

import lombok.AllArgsConstructor;
import lombok.Getter;
import lombok.Setter;
import lombok.ToString;

@ToString
@AllArgsConstructor
public class Livro {
	@Getter @Setter private UUID id;
	@Getter @Setter private String titulo;
	@Getter @Setter private String autor;
}
\end{lstlisting}

Graças ao pacote Lombok eliminamos a necessidade de criar métodos padrões Gets/Sets, o construtor com os argumentos e método toString. Veja mais detalhes sobre esse pacote em: \url{https://projectlombok.org/}.

\subsection{Classe Conexão}
Agora vamos criar o escopo de uma classe que realiza ações básicas com o Banco de dados:
\begin{lstlisting}[]
package meucass;

import java.util.ArrayList;
import java.util.List;
import java.util.stream.Collectors;
import com.datastax.driver.core.Cluster;
import com.datastax.driver.core.Cluster.Builder;
import com.datastax.driver.core.ResultSet;
import com.datastax.driver.core.Session;

public class CassandraConnector {

 private Cluster cluster;
 private Session session;
 private final String KEY_SPACE;
 private final String TABLE_NAME;
	
 public CassandraConnector(String keyspaceName, String table) {
  KEY_SPACE = keyspaceName;
  TABLE_NAME = table;
 }

 public boolean conexao(String node) {
  try {
   Builder b = Cluster.builder().addContactPoint(node);
   b.withPort(9042);
   cluster = b.build();
   session = cluster.connect();
   return true;
  } catch (Exception e) {
   return false;
  }
 }
	
 public void fechar() {
  session.close();
  cluster.close();
 }
	
 public boolean criarKeyspace(String replicationStrategy, int replicationFactor) {
  String query = "CREATE KEYSPACE IF NOT EXISTS " + KEY_SPACE + " WITH replication = {'class':'" + replicationStrategy + "','replication_factor':" + replicationFactor + "};";
  session.execute(query);
  ResultSet result = session.execute("SELECT * FROM system_schema.keyspaces;");
  List<String> matchedKeyspaces = result.all().stream().filter(r -> r.getString(0).equals(KEY_SPACE)).map(r -> r.getString(0)).collect(Collectors.toList());
  return matchedKeyspaces.size() == 1;
 }

 public void criarTabela() {
  String query = "CREATE TABLE IF NOT EXISTS " + KEY_SPACE + "." + TABLE_NAME + "(id uuid PRIMARY KEY, titulo text, autor text);";
  session.execute(query);
 }

 public void inserir(Livro livro) {
  String query = "INSERT INTO " + KEY_SPACE + "." + TABLE_NAME + "(id, titulo, autor) VALUES (" + livro.getId() + ", '" + livro.getTitulo() + "', '" + livro.getAutor() + "');";
  session.execute(query);
 }
	
 public List<livro> getAll() {
  String query = "SELECT * FROM " + KEY_SPACE + "." + TABLE_NAME;
  ResultSet rs = session.execute(query);
  List<Livro> livros = new ArrayList<>();
  rs.forEach(r -> {
   livros.add(new Livro(r.getUUID("id"), r.getString("titulo"), r.getString("autor")));
  });
  return livros;
 }

 public void eliminarTabela() {
  String query = "DROP TABLE IF EXISTS " + KEY_SPACE + "." + TABLE_NAME;
  session.execute(query);
 }

 public void eliminarKeyspace() {
  String query = "DROP KEYSPACE " + KEY_SPACE;
  session.execute(query);
 }
}
\end{lstlisting}

O método construtor recebe duas variáveis o nome da \textit{KeySpace} e Tabela. O método conexao() realiza a conexão com o banco a partir de um node e o valor da porta, cria uma sessão de conexão (variável \textit{session}) retorna um lógico se conseguiu ou não realizar essa conexão. O método fechar() encerra a conexão com a sessão e o cluster.

Para os métodos de ações temos: criarKeyspace() responsável por criar a \textit{KeySpace}. criarTabela() que cria a estrutura da nossa tabela. inserir() que a partir de um objeto livro adiciona este na tabela. getAll() retorna uma lista de livros que foram inseridos. eliminarTabela() e eliminarKeyspace() removem a tabela e \textit{KeySpace} respectivamente.

Se pararmos um pouco para pensar, veremos que tirando o método conexão todos os outros se comportam como qualquer outro driver de JDBC realizando as ações padrões como se estivéssemos usando um banco Postgres ou MySQL.

\subsection{Classe Teste}
Por fim vamos criar uma classe que testa toda essa conexão:
\begin{lstlisting}[]
package meucass;

import java.util.List;
import com.datastax.driver.core.utils.UUIDs;

public class Principal {
 private final String keyspaceName = "livraria";
 private CassandraConnector cc;

 public static void main(String[] args) {
  new Principal().executar();
 }

 private void executar() {
  cc = new CassandraConnector(keyspaceName, "livro");
  if (cc.conexao("localhost")) {
   passo1();
   passo2();
   passo3();
   passo4();
   passoFatal();
   cc.close();
  }
 }

 private void passo1() {
  System.out.println("Criar o KeySpace");
  if (cc.criarKeyspace("SimpleStrategy", 1)) {
   System.out.println("KeySpace criado");
  }
 }

 private void passo2() {
  System.out.println("Criar a Tabela");
  cc.criarTabela();
 }

 private void passo3() {
  System.out.println("Adicionar Registros");
  cc.inserir(new Livro(UUIDs.timeBased(), "O Tempo e o Vento", "Érico Veríssimo"));
  cc.inserir(new Livro(UUIDs.timeBased(), "Mentiras que os Homens Contam", "Luis Fernando Veríssimo"));
  cc.inserir(new Livro(UUIDs.timeBased(), "Vidas Secas", "Graciliano Ramos"));
  cc.inserir(new Livro(UUIDs.timeBased(), "Auto da Compadecida", "Ariano Suassuna"));
 }

 private void passo4() {
  System.out.println("Mostrar Registros");
  List<Livro> livros = cc.getAll();
  for (Livro livro : livros) {
   System.out.println(livro);
  }
 }

 private void passoFatal() {
  cc.eliminarTabela();
  cc.eliminarKeyspace();
 }
}
\end{lstlisting}

Esta classe será a nossa principal, agora podemos nos divertir a vontade com esse banco, tente explorar melhor, criar os métodos para alterar, excluir ou mesmo trazer um determinado livro - a CQL não é muito diferente em relação a SQL tradicional. Lembre-se que a Programação Orientada a Objetos é uma metodologia e não uma linguagem, se pratica essa forma ao usarmos os princípios da Orientação a Objetos e aproveitar a qualidade de extensibilidade do código.
\section{Python}
Python é uma linguagem de programação de alto nível, interpretada a partir de um script, Orientada a Objetos e de tipagem dinâmica. Foi lançada por Guido van Rossum em 1991. Não pretendo nesta apostila COMPARAR essa linguagem com Java (espero que nunca o faça), fica claro que os comandos são bem mais fáceis porém essas linguagens possuem diferentes propósitos.

Todos os comandos descritos abaixo foi utilizado no JupyterLab \cite{jupyteroficial}, então basta abrir um Notebook e digitá-los em cada célula conforme se apresentam.

\subsection{Ações com o Banco de Dados}
Baixar o pacote necessário: \\
\codigo{ !pip install cassandra-driver}

Importar os pacotes necessários: \\
\codigo{from cassandra.cluster import Cluster \\
from cassandra.query import SimpleStatement}

Nos conectamos ao servidor desta forma: \\
\codigo{cluster = Cluster() \\ session = cluster.connect()}

A partir desse objeto \textit{session} podemos dar qualquer comando para o Cassandra. Criar uma KeySpace:
\codigo{session.execute("CREATE KEYSPACE usuarios \\ WITH replication=\{'class': 'SimpleStrategy', 'replication\_factor': 1\}")}

Usar uma KeySpace: \\
\codigo{session.execute('USE usuarios;')}

Criar uma tabela: \\
\codigo{session.execute('CREATE TABLE usuario (id int primary key, nome text, \\ creditos int)')}

Adicionar uma linha: \\
\codigo{session.execute('INSERT INTO usuario (nome, creditos, id) VALUES (\%s, \%s, \%s)', \\ ('Fernando Anselmo', 42, 1))}

Nos conectamos a uma coleção desta forma: \\
\codigo{rows = session.execute('SELECT id, nome, creditos FROM usuario') \\
for u in rows: \\
\phantom{x}\hspace{4pt} print(u.id, u.nome, u.creditos)}

\subsection{Programa Completo}
Mas antes de encerramos realmente vejamos o seguinte programa completo em linguagem Python:
\begin{lstlisting}[]
from cassandra.cluster import Cluster
from cassandra.query import SimpleStatement
from random import randint

# Passo 1: Conectar ao Mongo
cluster = Cluster()
session = cluster.connect()

session.execute(
 "CREATE KEYSPACE negocios WITH replication={'class': 'SimpleStrategy', 'replication_factor': 1}")
session.execute('USE negocios;')

session.execute(
 'CREATE TABLE restaurante (id int PRIMARY KEY, nome text, nota int, cozinha text)')

# Passo 2: Criar Amostras de Dados
nomes = ['Kitchen', 'Espiritual', 'Mongo', 'Tastey', 'Big', 'Jr', 'Filho', 'City', 'Linux', 'Tubarão', 'Gado', 'Sagrado', 'Solo', 'Sumo', 'Lazy', 'Fun', 'Prazer', 'Gula']
tipo_emp = ['LLC', 'Inc', 'Cia', 'Corp.']
tipo_coz = ['Pizza', 'Bar', 'Fast Food', 'Italiana', 'Mexicana', 'Americana', 'Sushi', 'Vegetariana', 'Churrascaria']

for x in range(1, 501):
  nome = nomes[randint(0, (len(nomes)-1))] + ' ' + nomes[randint(0, (len(nomes)-1))] + ' ' + tipo_emp[randint(0, (len(tipo_emp)-1))]
  result = session.execute('INSERT INTO restaurante (id, nome, nota, cozinha) VALUES (%s, %s, %s, %s)', (x, nome, randint(1, 5), tipo_coz[randint(0, (len(tipo_coz)-1))]))

  # Passo 4: Mostrar no console o Object ID do Documento
  print('Criado {0} de 500 como {1}'.format(x, result))

  # Passo 5: Mostrar mensagem final
  print('500 Novos Negócios Culinários foram criados...')
\end{lstlisting}

O programa está auto-documentado e criar uma base com 500 registros.

\section{Conclusão}
Penso que depois dessa apostila, será possível iniciar a descobrir o Apache Cassandra para seus trabalhos, pois como vimos é bem fácil realizar os passos nesse banco e pouco importa a linguagem de programação. Não busquei nesta mostrar um exemplo mais completo para não limitar suas pesquisas e devemos considerar esta apenas como um pontapé inicial (\textit{KickStart}) para seus projetos.

Como visto este banco de dados pode ser facilmente utilizado com aplicações em linguagem Java ou gerar os modelos para \textit{Machine Learning} com Python e ainda colher o benefício de substituir os bancos de dados relacionais para grandes quantidades de dados, sendo que esta é a grande motivação para NoSQL como forma de resolver o problema de escalabilidade dos bancos tradicionais.

As principais linguagens de programação possuem suporte e aqui vimos apenas Java e Python, porém existem muitas outras como PHP, C, C++, C\#, JavaScript, Node.js, Ruby, R e Go. Esta apostila faz parte da série dos quatro tipos para Bancos de Dados no padrão NoSQL que estou tentando desmistificar e torná-los mais acessíveis tanto para as comunidades de Java e Python voltada especificamente para desenvolvedores ou cientistas de dados.
\begin{figure}[H]
	\centering
	\includegraphics[width=0.8\textwidth]{../sty/NoSQL}
	\caption{Tipos de Bancos de Dados}
\end{figure}

Sou um entusiasta do mundo \textbf{Open Source} e novas tecnologias. Qual a diferença entre Livre e Open Source? \underline{Livre} significa que esta apostila é gratuita e pode ser compartilhada a vontade. \underline{Open Source} além de livre todos os arquivos que permitem a geração desta (chamados de arquivos fontes) devem ser disponibilizados para que qualquer pessoa possa modificar ao seu prazer, gerar novas, complementar ou fazer o que quiser. Os fontes da apostila (que foi produzida com o LaTex) está disponibilizado no GitHub \cite{github}. Veja ainda outros artigos que publico sobre tecnologia através do meu Blog Oficial \cite{fernandoanselmo}.

%-----------------------------------------------------------------------------
% REFERÊNCIAS
%-----------------------------------------------------------------------------
\begin{thebibliography}{8}
  \bibitem{cassandraoficial} 
  Página do Banco Apache Cassandra \\
  \url{https://cassandra.apache.org}

  \bibitem{javaoficial} 
  Página do Oracle Java \\
  \url{http://www.oracle.com/technetwork/java}
  
  \bibitem{pythonoficial} 
  Página do Python \\
  \url{https://www.python.org}

  \bibitem{sts} 
  Editor Spring Tool Suite para códigos Java \\
  \url{https://spring.io/tools}

  \bibitem{jupyteroficial} 
  Página do Jupyter \\
  \url{https://jupyter.org}

  	\bibitem{fernandoanselmo} 
	Fernando Anselmo - Blog Oficial de Tecnologia \\
	\url{http://www.fernandoanselmo.blogspot.com.br/}
	
	\bibitem{publicacao} 
	Encontre essa e outras publicações em \\
	\url{https://cetrex.academia.edu/FernandoAnselmo}
	
	\bibitem{github} 
	Repositório para os fontes da apostila \\
	\url{https://github.com/fernandoans/publicacoes}
\end{thebibliography}
  
\end{document}
