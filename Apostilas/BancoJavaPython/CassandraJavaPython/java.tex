\section{Linguagem Java}
Java é considerada a linguagem de programação orientada a objetos mais utilizada no Mundo, é a base para construção de ferramentas como Hadoop, Pentaho, Weka e muitas outras utilizados comercialmente. Foi desenvolvida na década de 90 por uma equipe de programadores chefiada por \textit{James Gosling} para o projeto Green, na empresa Sun Microsystems - tornou-se nessa época como a linguagem que os programadores mais baixaram e o sucesso foi instantâneo. Em 2008 o Java foi adquirido pela Oracle Corporation.

\subsection{Driver JDBC de Conexão}
Para proceder a conexão com Java, é necessário baixar um driver JDBC (Java Database Connection). Existem vários drivers construídos, porém o driver oficialmente suportado pelo Apache Cassandra se encontra no endereço: \url{https://docs.datastax.com/en/driver-matrix/doc/driver_matrix/common/driverMatrix.html}

Para utilizar o driver é necessário criar um projeto (Utilizaremos o \textbf{Spring Tool Suite 4}, porém pode utilizar qualquer outro editor de sua preferência).

No STS4 acessar a seguinte opção no menu: File $\triangleright$ New $\triangleright$ Java Project. Informar o nome do projeto (meucass), não esquecer de modificar a opção "Use an environment JRE" para a versão correta da Java Runtime desejada e pressionar o botão Finish. Se está tudo correto teremos a seguinte situação na aba \textit{Project Explorer}.

Vamos convertê-lo para um projeto Apache Maven. Clicar com o botão direito do mouse no projeto e acessar a opção: Configure $\triangleright$ Convert to Maven Project. Na janela apenas pressione o botão \textit{Finish}. Se tudo está correto observamos que o projeto ganhou uma letra \textbf{M} o que indica agora é um projeto padrão Maven. Então foi criado um arquivo chamado \textbf{pom.xml}. 

Acessar este arquivo e antes da tag BUILD, inserir a tag DEPENDENCIES:
\begin{lstlisting}[]
<dependencies>
 <dependency>
  <groupId>com.datastax.cassandra</groupId>
  <artifactId>cassandra-driver-core</artifactId>
  <version>3.1.0</version>
 </dependency>
 <dependency>
  <groupId>org.cassandraunit</groupId>
  <artifactId>cassandra-unit</artifactId>
  <version>3.0.0.1</version>
 </dependency>
 <dependency>
  <groupId>org.projectlombok</groupId>
  <artifactId>lombok</artifactId>
  <scope>provided</scope>
  <version>1.18.20</version>
 </dependency>
</dependencies>
\end{lstlisting}

Observamos que na pasta \textbf{Maven Dependencias} foi baixado a versão 3.1.0 do driver JDBC do Cassandra.

\subsection{Classe Livro}
Estamos prontos para testarmos a conexão entre porém vamos criar uma classe chamada \textbf{Livro} no pacote \textbf{meucass} e inserir nesta a seguinte codificação:
\begin{lstlisting}[]
package meucass;

import java.util.UUID;

import lombok.AllArgsConstructor;
import lombok.Getter;
import lombok.Setter;
import lombok.ToString;

@ToString
@AllArgsConstructor
public class Livro {
	@Getter @Setter private UUID id;
	@Getter @Setter private String titulo;
	@Getter @Setter private String autor;
}
\end{lstlisting}

Graças ao pacote Lombok eliminamos a necessidade de criar métodos padrões Gets/Sets, o construtor com os argumentos e método toString. Veja mais detalhes sobre esse pacote em: \url{https://projectlombok.org/}.

\subsection{Classe Conexão}
Agora vamos criar o escopo de uma classe que realiza ações básicas com o Banco de dados:
\begin{lstlisting}[]
package meucass;

import java.util.ArrayList;
import java.util.List;
import java.util.stream.Collectors;
import com.datastax.driver.core.Cluster;
import com.datastax.driver.core.Cluster.Builder;
import com.datastax.driver.core.ResultSet;
import com.datastax.driver.core.Session;

public class CassandraConnector {

 private Cluster cluster;
 private Session session;
 private final String KEY_SPACE;
 private final String TABLE_NAME;
	
 public CassandraConnector(String keyspaceName, String table) {
  KEY_SPACE = keyspaceName;
  TABLE_NAME = table;
 }

 public boolean conexao(String node) {
  try {
   Builder b = Cluster.builder().addContactPoint(node);
   b.withPort(9042);
   cluster = b.build();
   session = cluster.connect();
   return true;
  } catch (Exception e) {
   return false;
  }
 }
	
 public void fechar() {
  session.close();
  cluster.close();
 }
	
 public boolean criarKeyspace(String replicationStrategy, int replicationFactor) {
  String query = "CREATE KEYSPACE IF NOT EXISTS " + KEY_SPACE + " WITH replication = {'class':'" + replicationStrategy + "','replication_factor':" + replicationFactor + "};";
  session.execute(query);
  ResultSet result = session.execute("SELECT * FROM system_schema.keyspaces;");
  List<String> matchedKeyspaces = result.all().stream().filter(r -> r.getString(0).equals(KEY_SPACE)).map(r -> r.getString(0)).collect(Collectors.toList());
  return matchedKeyspaces.size() == 1;
 }

 public void criarTabela() {
  String query = "CREATE TABLE IF NOT EXISTS " + KEY_SPACE + "." + TABLE_NAME + "(id uuid PRIMARY KEY, titulo text, autor text);";
  session.execute(query);
 }

 public void inserir(Livro livro) {
  String query = "INSERT INTO " + KEY_SPACE + "." + TABLE_NAME + "(id, titulo, autor) VALUES (" + livro.getId() + ", '" + livro.getTitulo() + "', '" + livro.getAutor() + "');";
  session.execute(query);
 }
	
 public List<livro> getAll() {
  String query = "SELECT * FROM " + KEY_SPACE + "." + TABLE_NAME;
  ResultSet rs = session.execute(query);
  List<Livro> livros = new ArrayList<>();
  rs.forEach(r -> {
   livros.add(new Livro(r.getUUID("id"), r.getString("titulo"), r.getString("autor")));
  });
  return livros;
 }

 public void eliminarTabela() {
  String query = "DROP TABLE IF EXISTS " + KEY_SPACE + "." + TABLE_NAME;
  session.execute(query);
 }

 public void eliminarKeyspace() {
  String query = "DROP KEYSPACE " + KEY_SPACE;
  session.execute(query);
 }
}
\end{lstlisting}

O método construtor recebe duas variáveis o nome da \textit{KeySpace} e Tabela. O método conexao() realiza a conexão com o banco a partir de um node e o valor da porta, cria uma sessão de conexão (variável \textit{session}) retorna um lógico se conseguiu ou não realizar essa conexão. O método fechar() encerra a conexão com a sessão e o cluster.

Para os métodos de ações temos: criarKeyspace() responsável por criar a \textit{KeySpace}. criarTabela() que cria a estrutura da nossa tabela. inserir() que a partir de um objeto livro adiciona este na tabela. getAll() retorna uma lista de livros que foram inseridos. eliminarTabela() e eliminarKeyspace() removem a tabela e \textit{KeySpace} respectivamente.

Se pararmos um pouco para pensar, veremos que tirando o método conexão todos os outros se comportam como qualquer outro driver de JDBC realizando as ações padrões como se estivéssemos usando um banco Postgres ou MySQL.

\subsection{Classe Teste}
Por fim vamos criar uma classe que testa toda essa conexão:
\begin{lstlisting}[]
package meucass;

import java.util.List;
import com.datastax.driver.core.utils.UUIDs;

public class Principal {
 private final String keyspaceName = "livraria";
 private CassandraConnector cc;

 public static void main(String[] args) {
  new Principal().executar();
 }

 private void executar() {
  cc = new CassandraConnector(keyspaceName, "livro");
  if (cc.conexao("localhost")) {
   passo1();
   passo2();
   passo3();
   passo4();
   passoFatal();
   cc.close();
  }
 }

 private void passo1() {
  System.out.println("Criar o KeySpace");
  if (cc.criarKeyspace("SimpleStrategy", 1)) {
   System.out.println("KeySpace criado");
  }
 }

 private void passo2() {
  System.out.println("Criar a Tabela");
  cc.criarTabela();
 }

 private void passo3() {
  System.out.println("Adicionar Registros");
  cc.inserir(new Livro(UUIDs.timeBased(), "O Tempo e o Vento", "Érico Veríssimo"));
  cc.inserir(new Livro(UUIDs.timeBased(), "Mentiras que os Homens Contam", "Luis Fernando Veríssimo"));
  cc.inserir(new Livro(UUIDs.timeBased(), "Vidas Secas", "Graciliano Ramos"));
  cc.inserir(new Livro(UUIDs.timeBased(), "Auto da Compadecida", "Ariano Suassuna"));
 }

 private void passo4() {
  System.out.println("Mostrar Registros");
  List<Livro> livros = cc.getAll();
  for (Livro livro : livros) {
   System.out.println(livro);
  }
 }

 private void passoFatal() {
  cc.eliminarTabela();
  cc.eliminarKeyspace();
 }
}
\end{lstlisting}

Esta classe será a nossa principal, agora podemos nos divertir a vontade com esse banco, tente explorar melhor, criar os métodos para alterar, excluir ou mesmo trazer um determinado livro - a CQL não é muito diferente em relação a SQL tradicional. Lembre-se que a Programação Orientada a Objetos é uma metodologia e não uma linguagem, se pratica essa forma ao usarmos os princípios da Orientação a Objetos e aproveitar a qualidade de extensibilidade do código.