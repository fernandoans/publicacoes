\section{Python}
Python é uma linguagem de programação de alto nível, interpretada a partir de um script, Orientada a Objetos e de tipagem dinâmica. Foi lançada por Guido van Rossum em 1991. Não pretendo nesta apostila COMPARAR essa linguagem com Java (espero que nunca o faça), fica claro que os comandos são bem mais fáceis porém essas linguagens possuem diferentes propósitos.

Todos os comandos descritos abaixo foi utilizado no JupyterLab \cite{jupyteroficial}, então basta abrir um Notebook e digitá-los em cada célula conforme se apresentam.

\subsection{Pacote Necessário}
Baixar o pacote necessário: \\
\codigo{!pip install neo4j}

Importar os pacotes necessários: \\
\codigo{from neo4j import GraphDatabase}

Para realizarmos a manipulação dos dados a melhor forma é criarmos uma classe que terá a seguinte estrutura:
\begin{lstlisting}[]
class Teste:
  def __init__(self, uri, usuario, senha):
    try:
      self.driver = GraphDatabase.driver(uri, auth=(usuario, senha))
    except Exception as e:
      print("Falhou para acessar o Driver:", e)

  def close(self):
    self.driver.close()
\end{lstlisting}	

Nossa classe Teste possui o construtor que recebe URI (endereço), usuário e a senha de acesso ao banco e realiza essa conexão através da classe \textit{GraphDatabase}. Ao encerrar o objeto a conexão é encerrada.
Implementamos a chamada principal:
\begin{lstlisting}[]
if __name__ == "__main__":
  app = Teste("bolt://localhost:7687", "neo4j", "test")
  app.close()
\end{lstlisting}

Agora podemos testar a vontade sabendo que existem 2 métodos principais: \textit{write\_transaction} que nos permite realizar as operações de gravação na base de dados e \textit{read\_transaction} para realizarmos as transações de consulta. Vamos começar inserindo as pessoas na base, para tanto na classe abaixo do método \textit{close()}, inserimos a seguinte codificação:
\begin{lstlisting}[]
  def criarPessoa(self, nome_pessoa, modelo):
    with self.driver.session() as session:
      session.write_transaction(
        self._criarPessoa, nome_pessoa, modelo)

  @staticmethod
  def _criarPessoa(tx, nome_pessoa, modelo):
    tx.run("CREATE (:Pessoa {nome: $n1, modelo_carteira: $m1})", n1=nome_pessoa, m1=modelo)	
\end{lstlisting}	

O primeiro método abre uma seção com o \textit{Driver} em seguida uma transação de escrita e chama o segundo método para proceder a inclusão do registro em linguagem \textit{Cypher}. Para executar na chamada principal entre a criação do objeto \textit{app} e seu fechamento, inserimos:
\begin{lstlisting}[]
  app.criarPessoa('Fernando', 'AB')
  app.criarPessoa('Anselmo', 'B')
\end{lstlisting}	

Executamos e comentamos essas chamadas pois se rodarmos novamente criará mais objetos. Para testarmos de forma simples a leitura criamos mais 2 métodos na classe:
\begin{lstlisting}[]
  def obterPessoas(self):
    with self.driver.session() as session:
      result = session.read_transaction(self._obterPessoas)
      return result

  @staticmethod
  def _obterPessoas(tx):
    result = tx.run("MATCH (p:Pessoa) RETURN p.nome AS nome")
    return [record["nome"] for record in result]
\end{lstlisting}	

E inserimos a chamada deste no ponto que comentamos as chamadas anteriores:
\begin{lstlisting}[]
  print(app.obterPessoas())	
\end{lstlisting}	

E como resposta teremos a lista de pessoas registradas: \\
\codigo{['Anselmo', 'Fernando']}

\subsection{Sem utilizar uma Estrutura de Classe}
Não necessitamos utilizar uma estrutura de classe para criarmos nossas transações basta que elas sejam executadas através de uma seção única, vamos criar um exemplo mais completo para demonstrar isso:
\begin{lstlisting}[]
from neo4j import GraphDatabase

graphDB_Driver = GraphDatabase.driver("bolt://localhost:7687", auth=("neo4j", "test"))

# Consultas
cqlUniversidades = "MATCH (u:universidade) RETURN u.nome"
cqlDistancia = "MATCH (p:professor {nome:'Fernando Anselmo'})-[r]->(u:universidade) RETURN u.nome,r.km"

# Criação dos registros
cqlCriar = """CREATE (p:professor { nome: "Fernando Anselmo"}),
(s:universidade {nome: "IESB Campus Sul"}),
(n:universidade {nome: "IESB Campus Norte"}),
(c:universidade {nome: "IESB Campus Ceilândia"}),
(p)-[:PERCORRE {km: 8}]->(s),
(p)-[:PERCORRE {km: 1}]->(n),
(p)-[:PERCORRE {km: 18}]->(c)"""

with graphDB_Driver.session() as secao:
  secao.run(cqlCriar)
  nodes = secao.run(cqlUniversidades)

  print("Lista dos Campus do IESB:")
  for node in nodes:
    print(node["u.nome"])

  nodes = secao.run(cqlDistancia)
  print("\nDistância Percorrida:")
  for node in nodes:
    print("Para ir ao", node["u.nome"], "percorre", node["r.km"], "Km")
\end{lstlisting}	

Criamos a conexão com o banco, e definimos 3 variáveis (apenas para um melhor entendimento): \textbf{cqlUniversidades} que mostra os campus criados, \textbf{cqlDistancia} que mostra a distância percorrida por um determinado professor e \textbf{cqlCriar} que cria todos os registros. Abrimos uma seção com \textbf{with} e podemos rodar qualquer comando para obter os resultados desejados. E como saída teremos: \\
\codigo{Lista dos Campus do IESB:} \\
\codigo{IESB Campus Norte} \\
\codigo{IESB Campus Sul} \\
\codigo{IESB Campus Ceilândia}

\codigo{Distância Percorrida:} \\
\codigo{Para ir ao IESB Campus Norte percorre 1 Km} \\
\codigo{Para ir ao IESB Campus Sul percorre 8 Km} \\
\codigo{Para ir ao IESB Campus Ceilândia percorre 18 Km}

Não pretendo de modo nenhum, nessa apostila, repetir mais do mesmo, sendo assim já temos o básico para realizar todos os exemplos que foram mostrados anteriormente. Se deseja ir além existem excelentes tutoriais na Internet e entre eles recomendo: \vspace{-1em}
\begin{itemize}
	\item \textbf{Create a graph database in Neo4j using Python} de CJ Sulivan \\ \url{https://towardsdatascience.com/create-a-graph-database-in-neo4j} \\ 
	\url{-using-python-4172d40f89c4}
	\item \textbf{Build a Restaurant Recommendation Engine Using Neo4j} de Ng Wai Foong \\ \url{https://betterprogramming.pub/build-a-restaurant-recommendation-engine} \\
	\url{-using-neo4j-9d13ebdd4736}
	\item \textbf{Neo4j Movies Application} do  Github de Neoj-examples \\
	\url{https://github.com/neo4j-examples/movies-python-bolt}
	\item \textbf{Building a Career Recommendation Engine With Neo4j} de Otavio Santana \\ \url{https://dzone.com/articles/when-neo4j-faces-java-ee-ops-ee4j}
\end{itemize}
