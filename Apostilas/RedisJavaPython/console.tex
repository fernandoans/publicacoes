\section{Redis-Cli}
Diferente dos outros bancos o Redis preza por ser extremamente simples, podemos selecionar qualquer base com o seguinte comando: \\
\codigo{> select N}

Onde este N é simplesmente o número da base, a primeira inicia com 0, sendo assim: \\
\codigo{> select 0}

Este comando verifica a quantidade de elementos: \\
\codigo{> dbsize}

Outro simples comando para saber que está tudo OK é: \\
\codigo{> ping}

E a resposta deve ser \textbf{PONG} o que significa "estamos sem problemas". Para inserirmos um simples elemento: \\
\codigo{> set valor 100}

E para resgatá-lo: \\
\codigo{> get valor}

Qualquer tipo para o Redis pode ser tratado como uma String, assim para obtermos o tamanho de uma determinada chave: \\
\codigo{> strlen valor}

Mas observamos também que por ser um número podemos realizar operações como incrementar o valor: \\
\codigo{> incr valor}

Ou decrementar o valor: \\
\codigo{> decr valor}

O mais curioso é que podemos ainda concatenar (anexar) mais informações: \\
\codigo{> decr 200}

E agora nossa chave tem o valor "100200". E podemos adicionar mais valores: \\
\codigo{> incrby valor 50}

Ou diminuir valores: \\
\codigo{> decrby valor 50}

\subsection{Chave-Valor}
O maior uso do Redis e o armazenamento de informações como uma tabela no formato chaves-valores: \\
\codigo{> hmset identificacao nome Fernando cargo Analista materia Estatística}

Adicionar mais uma propriedade: \\
\codigo{> hset identificacao cidade Brasília}

Recuperar uma determinada propriedade: \\
\codigo{> hmget identificacao cargo}

Obter todos os valores: \\
\codigo{> hvals identificacao}

Obter todos os nomes das chaves: \\
\codigo{> hkeys identificacao}

Ou mais completo como: \\
\codigo{> hgetall identificacao}

Obter o tamanho: \\
\codigo{> hlen identificacao}

Eliminar uma propriedade: \\
\codigo{> hdel identificacao materia}

\subsection{Listas de valores}
Para criar uma chave como valores de lista e adicionar elementos a ela (a direita ou no final): \\
\codigo{> rpush tipos persa ragdoll bengali}

Ou então (a esquerda ou no início): \\
\codigo{> lpush tipos siamês}

Recuperar todos os tipos (se ainda não percebeu de gatos): \\
\codigo{> lrange tipos 0 -1}

Ou um determinado tipo (por exemplo o terceiro elemento - a lista inicia do 0): \\
\codigo{> lrange tipos 2 2}

Verificar o tamanho da lista: \\
\codigo{> llen tipos}

Eliminar determinado elemento: \\
\codigo{> lpop tipos 2}

\subsection{Conjuntos de valores}
Para criarmos uma chave como valores de conjuntos e adicionar elementos a ela: \\
\codigo{> sadd paises Brasil EUA Portugal Espanha}

Recuperar a quantidade de países: \\
\codigo{> scard paises}

Recuperamos todos os países: \\
\codigo{> smembers paises}

Ou obter um país aleatoriamente: \\
\codigo{> srandmember paises 1}

Eliminar determinado elemento: \\
\codigo{> srem paises 2}

\subsection{Movimentos e Exclusão}
Vericar se uma determinada chave existe: \\
\codigo{> exists paises}

Mover uma determinada chave para outro banco: \\
\codigo{> move tipos 1}

Modificar o nome de uma determinada chave: \\
\codigo{> rename paises destinos}

Para excluirmos qualquer chave temos mais um comando extremamente simples: \\
\codigo{> del destinos}