%------------------------------------------------------------------------------------
%	CHAPTER 2
%------------------------------------------------------------------------------------
\chapterimage{headMontagem.png}
\chapter{Primeiros Programas}

\begin{remark}
	"Sistemas antigos não são um fardo; são a base do progresso — refatorá-los é um ato de respeito pelo passado e compromisso com o futuro." (\textit{Sandi Metz}, especialista em design de software.) 
\end{remark}

\section{Hello World}\index{Primeiros Programas}
Escrever um programa 'Hello World' é uma forma simples e prática de verificar se o ambiente de desenvolvimento está configurado corretamente, e garantir que possamos começar a programar sem obstáculos técnicos. Geralmente e tradicionalmente este é o primeiro passo no aprendizado de qualquer linguagem de programação, assim os iniciantes podem entender a estrutura básica do código e se familiarizar com sua sintaxe.

Primeiramente devemos compreender que o Cobol é uma linguagem que fortemente segue determinados princípios de programação, e qualquer desvio desses resultará em erro.
\begin{itemize}
	\item As colunas de 1 a 6 do nosso programa é reservada para numeração das linhas.
	\item A coluna 7 e reservada especialmente para continuação de linha ou início de comentário da linha.
	\item Nossa codificação sempre deve ser iniciada a partir da coluna 8, e o deslocamento de código é realizado por um TAB.
	\item TODA instrução deve terminar com um PONTO FINAL.
\end{itemize}

A estrutura do programa sempre deve conter 4 divisões (mesmo que não sejam utilizadas devem sempre aparecer):
\begin{itemize}
	\item \textbf{Identification Division} - É a primeira divisão do COBOL e contém informações básicas sobre o programa. Usada para identificá-lo, opcionalmente, o autor, a data de criação, e outros detalhes administrativos. O comando mais comum nessa divisão é o \textbf{PROGRAM-ID}, que nomeia o programa. É essencial para documentar o propósito do código, especialmente em sistemas onde múltiplos programas podem interagir.
	\item \textbf{Environment Division} - Define o ambiente em que o programa será executado, incluindo as especificações do hardware e software. Subdividida em duas seções principais:
	\begin{itemize}
		\item \textbf{Configuration Section}: Especifica detalhes sobre o sistema, como a máquina onde o programa será executado.
		\item \textbf{Input-Output Section}: Define os dispositivos de entrada e saída utilizados pelo programa, como arquivos, impressoras ou terminais. Essencial para garantir que o programa funcione corretamente no ambiente de produção.
	\end{itemize}	
	\item \textbf{Data Division} - Responsável pela definição de todas as variáveis e estruturas de dados usadas no programa. Além disso, permite que os dados sejam organizados de forma lógica, facilitando o acesso e a manipulação. Subdividida em várias seções:
	\begin{itemize}
		\item \textbf{File Section}: Descreve os arquivos usados pelo programa, incluindo sua estrutura e organização.
		\item \textbf{Working-Storage Section}: Declara variáveis que mantêm dados em memória durante a execução do programa.
		\item \textbf{Local-Storage Section}: Contém variáveis que são alocadas apenas durante a execução de um procedimento específico.
		\item \textbf{Linkage Section}: Define variáveis usadas para comunicar dados entre programas.
	\end{itemize}
	\item \textbf{Procedure Division} - É aqui onde a lógica do programa é implementada. Contém as instruções e os comandos que controlam o fluxo de execução. Os procedimentos são organizados em parágrafos e seções, o que permite modularizar o código para melhorar sua legibilidade e reutilização. É aqui que operações como leitura e escrita de arquivos, cálculos e controle de fluxo são realizadas. Essa divisão é o "coração" do programa, onde os objetivos específicos são alcançados.
\end{itemize}

E sem mais "delongas", aqui está o nosso programa completo:
\begin{lstlisting}[]
000001 IDENTIFICATION DIVISION.
000002     PROGRAM-ID. HELLOWORLD.
000003     AUTHOR. Fernando Anselmo.
000004
000005 ENVIRONMENT DIVISION.
000006       
000007 DATA DIVISION.
000008*Aqui inicia as ações do programa    
000009 PROCEDURE DIVISION.
000010 PRINCIPAL.
000011     DISPLAY "Hello World".
000012
000013     STOP RUN.
000014 END PROGRAM HELLOWORLD.

\end{lstlisting}

Necessariamente não precisamos inserir os números nas colunas de 1 a 6 para que o programa funcione, porém devemos iniciar sempre na coluna 8, observe que na linha 8 temos um comentário, este é sinalizado com um "*" na coluna 7.

As linhas de 1 a 3 são referentes a identificação do programa, este ao mínimo deve ter \textbf{PROGRAM-ID} que identifica o nome do programa e \textbf{AUTHOR} com o autor do programa. As linhas 5 e 6 são correspondentes as duas próximas divisões, não é necessário nenhuma informação, porém essas divisões devem estar contidas no programa. E por fim a partir da linha 9 inicia o programa propriamente dito.

Um programa sempre inicia com um "label", que pode ser qualquer um a escolha, colocamos \textbf{PRINCIPAL} aqui, do mesmo modo que poderíamos colocar \textbf{INICIAL} ou qualquer outro nome. Após esse, temos as ações que queremos executar, no caso o comando DISPLAY que mostra uma determinada informação na \textit{console}. Para terminar nossas execuções temos o comando \textbf{STOP RUN} (linha 13) e para finalizar o programa a instrução \textbf{END PROGRAM} com o nome do programa que foi definido pela \textbf{PROGRAM-ID} que marca explicitamente o fim do código.

\begin{note}[Linha em Branco]
	Por padrão o GNU Cobol, exige que ao final do programa SEMPRE exista uma linha em final em branco. Sendo assim lembre-se de ao final sempre deixar uma linha em BRANCO.
\end{note}

Salvamos este programa com o nome \textit{HelloWorld.cbl} (outra extensão que poderia ser usada é "cob"), e em seguida devemos compilar com o comando: \\
\codigo{\$ cobc -x HelloWorld.cbl}

Nesse momento aconteceu a magia pois o nosso programa foi transformado em um executável do Linux e basta apenas o comando: \\
\codigo{\$ ./HelloWorld}

E teremos nossa mensagem mostrada.

\section{Entrada de Dados}\index{Primeiros Programas}
Em qualquer linguagem de programação iniciamos com os processos de Entrada/Saída de dados, começamos pela saída, que em Cobol é obtida através do comando \textbf{DISPLAY} e agora vamos partir para a entrada, que em Cobol é obtida através do comando \textbf{ACCEPT}. Mas não é tão simples assim pois precisamos definir uma variável que receberá a informação do que vamos digitar.

A criação de variáveis em Cobol é realizada na \textbf{DATA DIVISION}, mais especificamente na seção \textbf{WORKING-STORAGE SECTION} que descreve os dados internos, nesse ponto podemos criar quaisquer variáveis que desejarmos, porém existem regras para isso, a partir da coluna 8 devemos iniciar um "\textit{level number}", esse número sempre inicia com "01", em seguida colocamos o nome da nossa variável, que será utilizada pelo programa, e o tipo de dado que será armazenado, como estamos tratando de variável, usamos a cláusula \textbf{PIC} (abreviatura de \textit{PICTURE}) e em seguida o tipo, que podem ser, por padrão:

\begin{table}[H]
	\centering 
	\begin{tabular}{c | l }
		\textbf{Símbolo} & \textbf{Significado} \\ \hline
		x & Campo alfanumérico \\
		9 & Campo numérico \\
		A & Campo alfabético \\
		V & Campo decimal assumido; usado apenas em campos numéricos \\
		S & Sinal operacional; usado apenas em campos numéricos \\
	\end{tabular}
\end{table}

E finalmente colocamos o tamanho do campo, mas vamos com calma e mostrar um exemplo na prática como isso funciona, por exemplo, ao invés de termos um programa estático "Hello World", desejamos que o usuário tenha a possibilidade de colocar seu nome e a data de seu nascimento, como fazemos isso:
\begin{lstlisting}[]
000001 IDENTIFICATION DIVISION.
000002     PROGRAM-ID. COMOVAI.
000003     AUTHOR. Fernando Anselmo.
000004
000005 ENVIRONMENT DIVISION.
000006       
000007 DATA DIVISION.
000008 WORKING-STORAGE SECTION.
000009
000010 01 NOME        PIC A(020).
000011
000012 01 DATA-ATUAL.
000013    05 ANO-ATUAL PIC 9(004).
000014    05 MES-ATUAL PIC 9(002).
000015    05 DIA-ATUAL PIC 9(002).
000016
000017 PROCEDURE DIVISION.
000018 PRINCIPAL.
000019     DISPLAY "Entre com seu Nome: ".
000020     ACCEPT NOME.
000021     ACCEPT DATA-ATUAL FROM DATE YYYYMMDD.
000022
000023     DISPLAY "Bem vindo " NOME.
000024     DISPLAY "Sabia que hoje é " DIA-ATUAL "/" MES-ATUAL "/" 
            ANO-ATUAL.
000025
000026     STOP RUN.
000027 END PROGRAM COMOVAI.

\end{lstlisting}

O programa é bem similar ao visto anteriormente e como  não pretendo ficar me repetindo, assim vamos nos ater aos detalhes modificados, na seção \textbf{WORKING-STORAGE SECTION} criamos uma variável (linha 10) chamada NOME que será alfabética e permite de 0 a 20 posições. Em seguida outra variável denominada DATA-ATUAL que contém uma subestrutura (assim o nível deve ser um número menor, por padrão usamos de 5 em 5).

No nosso programa propriamente dito (PROCEDURE DIVISION), na linha 20 existe a espera pelo usuário digitar o NOME, e assim que acontecer na linha 21 é obtida a data atual automaticamente (parâmetro FROM DATE), na linha 23 esse NOME digitado é mostrado e na linha 24 a data atual, como essa linha ultrapassa a coluna 73 devemos continuar a informação na linha abaixo, note que não usamos o caracter de continuação na coluna 7, esse é utilizado apenas quando for um texto que queremos dar continuidade.

Salvamos este programa com o nome \textit{ComoVai.cbl} (outra extensão que poderia ser usada é "cob"), e em seguida devemos compilar com o comando: \\
\codigo{\$ cobc -x ComoVai.cbl}

Nesse momento aconteceu a magia pois o nosso programa foi transformado em um executável do Linux e basta apenas o comando: \\
\codigo{\$ ./ComoVai}

Será solicitado para digitarmos nosso nome e em seguida mostra a mensagem com a data atual.

\section{Cálculos em Cobol}\index{Primeiros Programas}
O comando \textbf{COMPUTE} é utilizado para realizar operações aritméticas de forma direta e simplificada, permite cálculos matemáticos como soma, subtração, multiplicação, divisão, e até expressões mais complexas. Combina a funcionalidade dos operadores matemáticos (+, -, *, /, **) para calcular o valor de uma expressão e armazenar o resultado em uma variável destino. 

\textbf{COMPUTE} é especialmente útil por sua clareza pois elimina a necessidade de comandos adicionais, como \textbf{ADD}, \textbf{SUBTRACT}, ou \textbf{MULTIPLY}, em situações simples. O resultado da expressão é automaticamente ajustado para o formato definido na variável de destino, incluindo arredondamento quando necessário, de acordo com as especificações da cláusula PICTURE.

Vejamos um caso prático com esse comando através do cálculo do IMC, que mostra o Índice de Massa Corporal, através da seguinte fórmula: \\
\[
\text{IMC} = \frac{\text{PESO}}{\text{ALTURA}^2}
\]

Sendo que este PESO é fornecido em quilos enquanto que a ALTURA em metros. Então temos um problema aqui, pois o usuário deve digitar um valor decimal, para conseguirmos isso vamos utilizar o formato: \textbf{PIC ZZ9V99}. Que representa um número com 3 dígitos inteiros e 2 casas decimais (exemplo: 51.75) o carácter \textbf{Z} elimina os zeros a esquerda. E o \textbf{V} é o ponto decimal virtual, que não aparece no armazenamento, mas é considerado no cálculo.

\begin{note}[Ponto ou Vírgula decimal?]
	A maioria das linguagens derivam dos padrões da Língua Inglesa. Cobol não é nenhuma exceção, sendo assim todas as entradas e saídas são mostradas com o uso do ponto (e não vírgula) decimal. Não esqueça disso na hora de digitar a informação.
\end{note}

Acredito que agora estamos prontos para codificar nosso programa:
\begin{lstlisting}[]
000001 IDENTIFICATION DIVISION.
000002     PROGRAM-ID. IMC.
000003     AUTHOR. Fernando Anselmo.
000004
000005 ENVIRONMENT DIVISION.
000006
000007 DATA DIVISION.
000008 WORKING-STORAGE SECTION. 
000009 
000010 01 ALTURA    PIC 9V99.
000011 01 PESO      PIC 999.
000012 01 IMC_TOTAL PIC ZZ9.99.
000013
000014 PROCEDURE DIVISION.
000015 PRINCIPAL.
000016     DISPLAY "Entre com sua altura (em Mt): ".
000017     ACCEPT ALTURA.
000018     DISPLAY "Entre com seu peso (em Kg): ".
000019     ACCEPT PESO.
000020
000021     COMPUTE IMC_TOTAL = PESO / (ALTURA ** 2).
000022     DISPLAY "Seu IMC é " IMC_TOTAL.
000023
000024 STOP RUN.
000025 END PROGRAM IMC.

\end{lstlisting}

Vamos para os detalhes, na seção WORKING-STORAGE SECTION que declara as variáveis utilizadas durante a execução do programa. Temos:
\begin{itemize}
	\item \textbf{ALTURA}: com formato PIC 9V99 que representa um número com 1 dígito inteiro e 2 casas decimais (exemplo: 1.65).
	\item \textbf{PESO}: com formato PIC 999 que representa um número inteiro de até 3 dígitos (exemplo: 75).
	\item \textbf{IMC\_TOTAL}: com formato PIC ZZ9.99 que representa o resultado do IMC com até 3 dígitos antes do ponto decimal e 2 após. Lembrando que carácter \textbf{Z} suprime zeros à esquerda, então, ao invés de apresentarmos 030.32 será exibido como 30.32.
\end{itemize}

Agora vamos na divisão que representa nosso programa, adoto por um simples padrão pessoal o rótulo \textbf{PRINCIPAL} que é o ponto de entrada do programa (semelhante a \textit{main} em outras linguagens).

Temos a entrada de duas variáveis a altura em metros e o peso em quilogramas. O comando \textbf{ACCEPT} captura os valores fornecidos pelo usuário e armazena nas variáveis correspondentes (ALTURA e PESO).

Em seguida o comando \textbf{COMPUTE} realiza o cálculo do IMC, o símbolo \textbf{**} representa a operação de exponenciação (potenciação). Utilizada para calcular o valor de uma base elevada a uma potência (ou expoente).

\begin{note}[Ordem de precedência]
	Os parênteses adicionados no comando são usados somente para facilitar a clareza do código pois não são necessários, a exponenciação tem maior precedência do que multiplicação (*) e divisão (/). Mas a utilização de parênteses facilita a clareza e leitura de fórmulas, e não prejudica a performance do programa, assim é recomendável sua utilização.
\end{note}
 
E por fim, o comando \textbf{DISPLAY} mostra a mensagem com o valor do IMC calculado (formatado pela cláusula de edição ZZ9.99).

Salvamos este programa com o nome \textit{IMC.cbl}, e em seguida devemos compilar com o comando: \\
\codigo{\$ cobc -x IMC.cbl}

Nesse momento aconteceu a magia pois o nosso programa foi transformado em um executável do Linux e basta apenas o comando: \\
\codigo{\$ ./IMC}

Será solicitado para digitarmos a altura (por exemplo, 1.6) e em seguida o peso (por exemplo, 70) e mostra a mensagem com o IMC calculado (que pelo exemplo deve ser 27.34).
\clearpage