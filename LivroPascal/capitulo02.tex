%------------------------------------------------------------------------------------
%	CHAPTER 2
%------------------------------------------------------------------------------------
\chapterimage{headMontagem.png}
\chapter{Passos Iniciais}

\begin{remark}
	"Aprender uma linguagem como Pascal é mais do que aprender a programar; é aprender a pensar logicamente." (\textit{Donald Knuth}, cientista da computação.) 
\end{remark}

\section{Hello World}\index{Passos Iniciais}
É tradição no mundo da programação que devemos começar com o programa "Hello World" quando estamos iniciando o estudo de uma nova linguagem, e quem sou eu para quebrar essa tradição, assim, aqui está o famoso "Hello World" em Pascal:
\begin{lstlisting}[]
program hello;

begin
  WriteLn('Hello World!');
end.
\end{lstlisting}

Chega a ser extremamente simples comparando com outras linguagens, o programa sempre inicia com a palavra reservada \textbf{program} e seguida o nome deste, finalizamos a instrução com um ";" (ponto e vírgula). O corpo principal do programa está entre na palavra reservada \textbf{begin} e seu término \textbf{end} que dessa vez deve ser finalizado com um "." (ponto final). E agora temos o comando, ou função se preferir, \textbf{WriteLn} que mostra algo na tela, neste caso o literal 'Hello World!', e uma observação muito importante, não utilizamos ASPAS DUPLAS (como os acostumados das linguagens derivadas de \textbf{C}) mas ASPAS SIMPLES para indicar as literais.

Salvamos este programa com o nome \textit{Hello.pas} (o nome não precisa ser o mesmo da cláusula \textbf{program}), e em seguida devemos compilar com o comando: \\
\codigo{\$ fpc Hello.pas}

Nesse momento aconteceu a magia pois o nosso programa foi transformado em um executável do Linux e basta apenas o comando: \\
\codigo{\$ ./Hello}

\section{Obtendo valores}\index{Passos Iniciais}
Agora que já agradamos aos deuses da programação, vamos tentar algo mais elaborado, como obter a entrada de duas variáveis numéricas e proceder cálculos aritméticos.

\begin{lstlisting}[]
program calculadora;
var 
  a,b: integer;

begin
  write('Informe o valor de a: ');
  read(a);
  write('Informe o valor de b: ');
  read(b);
  writeln('A soma é: ', a+b);
  writeln('A subtração é: ', a-b);
  writeln('A divisão é: ', a/b);
  writeln('A multiplicação é: ', a*b);
end.
\end{lstlisting}

Na linguagem Pascal não se importa se escrevemos as funções em letras maiúsculas ou minúsculas (desde que escritas corretamente), assim use a forma que preferir para estas.

Após a definição do programa, temos a palavra chave \textbf{var} que define duas variáveis inteiras \textbf{a} e \textbf{b}. Primeiro solicitamos o valor da primeira, a diferença entre as funções \textbf{write()} e \textbf{writeln()} e que a primeira escreve o valor e para o cursos no mesmo lugar enquanto que a segunda salta para a próxima linha. A função \textbf{read()} aguarda até que o usuário informe um valor e pressione a tecla ENTER e coloca o valor na variável selecionada.

Neste ponto podemos ter um erro no programa, note que nossa variável foi definida com um valor inteiro, caso o usuário informe algo diferente recebemos um \textbf{Runtime error}.

Considerando que dois valores foram informados corretamente, mostramos a soma (+), subtração (-), divisão (/) e multiplicação (*) desses, utilizamos a "," (virgula) para separar o literal do resultado.

Salvamos este programa com o nome \textit{Hello.pas} (o nome não precisa ser o mesmo da cláusula \textbf{program}), e em seguida devemos compilar com o comando: \\
\codigo{\$ fpc Calculadora.pas}

Nesse momento aconteceu a magia pois o nosso programa foi transformado em um executável do Linux e basta apenas o comando: \\
\codigo{\$ ./Calculadora}

\section{Fórmula de Bháskara}\index{Passos Iniciais}
Esse programa resolve uma equação do segundo grau com a utilização da \textbf{Fórmula de Bhaskara}.
\begin{lstlisting}[]
program bhaskara;
var
  a, b, c: integer;
var
  delta: real;

begin
  write('Informe o valor da raiz a: ');
  read(a);
  write('Informe o valor da raiz b: ');
  read(b);
  write('Informe o valor da raiz c: ');
  read(c);

  delta := (b*b) - (4*a*c);

  writeln('Para a função ', a, 'x^2 + ', b, 'x + ', c, ' temos:');
  if delta < 0 then
    writeln('Não existem raízes reais.')
  else if delta = 0 then
    writeln('Existe apenas uma raiz real: ', -b / (2 * a) :5:2)
  else
    begin
      writeln('Delta: ', delta:5:2);
      writeln('x' : ', (-b + sqrt(delta)) / (2*a) :5:2);
      writeln('x'' : ', (-b - sqrt(delta)) / (2*a) :5:2);
    end;
end.	
\end{lstlisting}

Começamos o programa com a definição de dois grupos de variáveis:  \vspace{-1em}
\begin{itemize}
	\item a, b, c: São os coeficientes da equação do segundo grau $ax^2 + bx + c = 0$. Foram declarados como inteiros (integer).
	\item delta: É valor decimal (real). O delta é a parte da fórmula de Bháskara que determina se existem raízes reais.
\end{itemize}

A primeira parte, do programa propriamente dito, obtém os valores de \textbf{a}, \textbf{b} e \textbf{c}. Calculamos o valor de Delta, com base na seguinte fórmula:
\[
\Delta = b^2 - 4ac
\]
A variável \textbf{delta} (uma variável auxiliar) armazena esse cálculo. Uma vez que temos esse valor, podemos julgá-lo para verificar se a equação tem soluções reais: \vspace{-1em}
\begin{itemize}
	\item Se $\delta < 0$, não existem raízes reais (não há solução no conjunto dos números reais).
	\item Se $\delta = 0$, há uma única raiz real, pois a fórmula se reduz $a - b / 2a$.
\end{itemize}

As raízes são calculadas com a fórmula:
\[
  x = \frac{-b \pm \sqrt{\Delta}}{2a}
\]
Salvamos este programa com o nome \textit{Equacao.pas}, e em seguida devemos compilar com o comando: \\
\codigo{\$ fpc Equacao.pas}

E executamos com o comando: \\
\codigo{\$ ./Equacao}

O programa funciona perfeitamente porém não é assim que normalmente usamos uma linguagem estruturada, pois dessa forma não estaremos facilitando a manutenção. Vamos utilizar 2 componentes em Pascal. \vspace{-2em}
\begin{enumerate}
	\item \textbf{Função (function)} - Uma função retorna um valor. Ou seja, calcula algo e retorna um determinado resultado.
	\item \textbf{Procedimento (procedure)} - Um procedimento não retorna valores diretamente. Utilizado para executar ações, como exibir mensagens ou modificar variáveis.
\end{enumerate}

Estrutura de uma função:
\begin{lstlisting}[]
function NomeDaFuncao(parametros): TipoDeRetorno;
begin
  NomeDaFuncao := valor;  { Retorna um valor }
end;
\end{lstlisting}

Estrutura de um procedimento:
\begin{lstlisting}[]
procedure NomeDoProcedimento(parametros);
begin
  { Faz algo, mas não retorna nada }
end;	
\end{lstlisting}

Sabendo disso, agora podemos reescrever o mesmo programa para:
\begin{lstlisting}[]
program bhaskara;

uses math;  { Para usar a função sqrt() }

{ Função para ler um número inteiro com uma mensagem personalizada }
function lerNumero(msg: string): integer;
begin
  write(msg);
  readln(lerNumero);
end;

{ Função para calcular Delta }
function calcularDelta(a, b, c: integer): real;
begin
  calcularDelta := (b * b) - (4 * a * c);
end;

{ Procedimento para calcular e exibir as raízes, se existirem }
procedure calcularRaizes(a, b: integer; delta: real);
var
  raiz1, raiz2: real;
begin
  if delta < 0 then
    writeln('Não existem raízes reais.')
  else if delta = 0 then
    writeln('Existe apenas uma raiz real: ', -b / (2 * a) :5:2)
  else
    begin
      raiz1 := (-b + sqrt(delta)) / (2 * a);
      raiz2 := (-b - sqrt(delta)) / (2 * a);
      writeln('Delta: ', delta:5:2);
      writeln('x' : ', raiz1:5:2);
      writeln('x'' : ', raiz2:5:2);
    end;
end;

{ Procedimento para ler os coeficientes da equação }
procedure lerCoeficientes(var a, b, c: integer);
begin
  a := lerNumero('Informe o valor de a: ');
  b := lerNumero('Informe o valor de b: ');
  c := lerNumero('Informe o valor de c: ');
end;

{ Programa principal }
var
  a, b, c: integer;
  delta: real;
begin
  lerCoeficientes(a, b, c);
  delta := calcularDelta(a, b, c);
  writeln('Para a função ', a, 'x^2 + ', b, 'x + ', c, ' temos:');
  calcularRaizes(a, b, delta);
end.
\end{lstlisting}

Com isso posto, apesar do programa ter crescido muito, observamos uma melhor legibilidade pois o código principal ficou menor e mais fácil de entender. Além disso podemos reusar várias funções ou procedimentos em diversos outros programas.

\section{Menu de um Sistema}\index{Passos Iniciais}
Antigamente os sistemas eram projetados para serem sequenciais, isso é ficávamos presos dentro de uma estrutura escolhida não existia o conceito de Menus \textit{PullDown} ou \textit{PopUps} conhecidos em sistemas gráficos atuais (a vida do programador era bem mais simples).
\begin{lstlisting}[]
program meu_menu;
var
 opc: integer;
begin
 repeat
  writeln('------ MENU ------');
  writeln('1. Cadastro');
  writeln('2. Busca');
  writeln('3. Ajuda');
  writeln('4. Sair');

  write('Sua opção é: ');
  read(opc);

  if (opc = 1) then
   writeln('Opção cadastro....')
  else if (opc = 2) then
   writeln('Opção busca....')
  else if (opc = 3) then
   writeln('Mostrar ajuda....')
  else if (opc = 4) then
   writeln('Até a próxima....')
  else
   writeln('Opção inválida!');

 until (opc = 4);
end.	
\end{lstlisting}

O comando \textbf{REPEAT-UNTIL} gera um laço de repetição até que alguma coisa aconteça, em linguagem modernas pode ser comparado ao comando \textbf{DO-WHILE} (Faça determinada coisa enquanto tal valor lógico for verdadeira), porém a diferença é que no comando REPEAT-UNTIL as ações se repetem \textbf{até} tal valor lógico seja verdadeiro. Mas não é a mesma coisa? Vamos escrever o comando \textbf{DO-WHILE}, no caso em linguagem \textbf{Java}, para entendermos essa diferença:
\begin{lstlisting}[]
do {
	ações a repetir
} while (opc != 4);
\end{lstlisting}

Observe que no comando \textbf{DO-WHILE} a lógica proposta é a negativa do valor, enquanto o valor de OPC \textbf{não for} 4; já para o comando REPEAT:
\begin{lstlisting}[]
	repeat
		ações a repetir
	until (opc = 4);
\end{lstlisting}

E agora partimos da positividade do valor, \textbf{até} o valor de OPC ser 4. Isso permite, em termos lógicos, mais fácil entendimento da situação.

Outro detalhe que devemos observar é a estrutura dos comando \textbf{IF-ELSE}. Neste caso a ausência de pontos e vírgulas ao final das linhas. Isso se deve ao fato de o comando não ter terminado e só usamos ponto e vírgula, ao término do comando.

\section{Gravar Arquivos}\index{Passos Iniciais}
Pascal era muito utilizado para gerar arquivos textos, esse era um modo simples e prático, principalmente relativo ao seu entendimento.

Vamos supor um simples arquivo de clientes contendo ID, Nome e Idade, iniciamos o programa com a definição desse arquivo:
\begin{lstlisting}[]
program GravarRegistros;

{$mode objfpc}

type
  TCliente = record
    ID: Integer;
    Nome: String[50];
    Idade: Integer;
  end;	

var
  arquivo: Text;
  cliente: TCliente;
\end{lstlisting}

O comando \textbf{mode objfpc} no garante que estamos trabalhando no modo adequado. Definimos nosso arquivo através do comando \textbf{type} e em seguida informamos a estrutura do registro do arquivo. Na seção das variáveis criamos as respectivas que serão utilizadas pelo programa.

\begin{lstlisting}[]
begin
  Assign(arquivo, 'clientes.txt');
  Rewrite(arquivo);
  while True do
  begin
    Write('Digite o ID do cliente (ou -1 para sair): ');
    ReadLn(cliente.ID);
    if cliente.ID = -1 then
      break;
    Write('Digite o nome do cliente');
    ReadLn(cliente.Nome);
    Write('Digite a idade do cliente: ');
    ReadLn(cliente.Idade);
    WriteLn(arquivo, cliente.ID, ';', cliente.Nome, ';', cliente.Idade);
  end;
  Close(arquivo);
  WriteLn('Registros gravados com sucesso!".');
end.	
\end{lstlisting}

No programa iniciamos associando o nome do arquivo externo a nossa variável tipo \textbf{Text}, em seguida abrimos o arquivo para escrita (que cria ou sobrescreve caso já exista). Em um laço "eterno" (isso só é possível pois ocorrerá uma interrupção deste em determinada condição), solicitamos as informações ao usuário, caso o ID seja -1 saímos do laço (usamos o comando \textbf{break} para interrompê-lo). A função \textbf{WriteLn} grava uma linha do arquivo.

Ao término fechamos o arquivo e mostramos uma mensagem para o usuário.

\clearpage