%------------------------------------------------------------------------------------
%	CHAPTER 4
%------------------------------------------------------------------------------------
\chapterimage{headerCap.jpeg}
\chapter{Registradores de 64 Bits}

\begin{remark}
	Dê o poder ao homem, e descobrirá quem realmente ele é. (Maquiavel - Diplomata, Autor e Historiador Italiano) 
\end{remark}

\section{Retornamos ao Hello World}\index{Registradores de 64 Bits}
Podemos pensar o seguinte, até o momento para compilar nossos programas usamos 64 Bits, na instrução "nasm -f elf64", e isso é uma grande verdade, porém nossos registradores são todos de 32 bits, e isso causa um problema, pois não vimos nada de, por exemplo PILHAS, pois só funcionam em registradores de 64 bits. Assim devemos aprender a usá-los e a usufruir de seus benefícios.

Vamos lembrar da nossa tabela exposta no primeiro capítulo desse livro:
\begin{table}[H]
	\centering 
	\begin{tabular}{c | c | l }
		\textbf{64 bits} & \textbf{32 bits} & \textbf{Utilização} \\ \hline
		rax & eax & Valores que são retornados dos comandos em um registrador \\
		rbx & ebx & Registrador preservado. Cuidado ao utilizá-lo \\
		rcx & ecx & Uso livre como por exemplo contador \\
		rdx & edx & Uso livre em alguns comandos \\
		rsp & esp & Ponteiro de uma pilha \\
		rbp & ebp & Registrador preservado. Algumas vezes armazena ponteiros de pilhas \\
		rdi & edi & Na passagem de argumentos, contém a quantidade desses \\
		rsi & esi & Na passagem de argumentos, contém os argumentos em si \\
	\end{tabular}
\end{table}

A primeira coluna contém os registradores que veremos neste capítulo, a mudança básica é simples trocamos a letra "E" pela letra "R". No arquivo \textbf{makefile} já visto não precisaremos trocar uma única linha e por enquanto não usaremos nenhuma biblioteca, vamos começar limpos e entender quais são as mudanças necessárias. Começamos por criar um arquivo chamado \textbf{lerarquivo.asm} com a seguinte codificação:
\begin{lstlisting}[]
section .data
  mensagem: db "Hello World 64 bits!!!", 0xA
  tam: equ $- mensagem

section .text

global _start:

_start:
\end{lstlisting}	

Até o presente momento nenhuma mudança no nosso código (os dois pontos é um mero preciosismo da minha parte), porém:
\begin{lstlisting}[]
  mov rax, 0x1
  mov rdi, 0x1
  mov rsi, mensagem
  mov rdx, 0x17
  syscall
\end{lstlisting}

Agora começamos a verdadeira mudança, se bem lembramos a sequencia seria EAX, EBX, ECX e EDX, porém dois registradores são trocados, EAX muda realmente para RAX mas se reparou bem seu valor agora não é "0x4" como era, EBX agora vai mudar para RDI (e não RBX) e ECX para RSI (e não RCX como seria de se esperar) e finalmente RDX para RDX (como seria de se esperar). Além disso tudo, aparece a instrução "syscall" que procede a chamada do antigo "int 0x80".

Agora vamos as diferenças para encerrar o programa:
\begin{lstlisting}[]
  mov rax, 0x3C
  xor rdi, rdi
  syscall
\end{lstlisting}	

A primeira instrução uma simples mudança no endereçamento no qual ao invés de "0x1" agora usamos o "0x3C", zeramos o segundo registrador mas dessa vez utilizamos o comando XOR e finalmente uma chamada a "syscall" que fará o serviço de entregar o bloco de instruções ao computador.

E ao compilar e executar teremos corretamente a mensagem aparecendo no terminal: \\
{\ttfamily Hello World 64 bits!!!}

Com esse pequeno programa vemos que não se trata então de uma simples mudança de registradores mas quase reaprender novamente os comandos, mas não se preocupe apenas mantenha a mente afiada que veremos muito mais nesse capítulo.
