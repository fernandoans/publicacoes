%----------------------------------------------------------------------------------------
% PACOTES E OUTRAS CONFIGURAÇÕES
%----------------------------------------------------------------------------------------
\documentclass[11pt]{scrartcl}
%----------------------------------------------------------------------------------
% The Legrand Orange Book
% Structural Definitions File
% Version 2.0 (9/2/15)
%----------------------------------------------------------------------------------
% \documentclass[a4paper,11pt]{book} % Fonte do livro

%----------------------------------------------------------------------------------
%	VARIOUS REQUIRED PACKAGES AND CONFIGURATIONS
%----------------------------------------------------------------------------------
\usepackage[top=3cm,bottom=3cm,left=3cm,right=2cm,headsep=10pt,a4paper]{geometry} % Page margins
\usepackage{import}
\usepackage{lipsum} % Inserts dummy text
\usepackage[brazil]{babel} % Brazil language/hyphenation
\usepackage{graphicx} % Required for including pictures
\usepackage{wrapfig} % Imagens que se cruzam com texto
\usepackage{tikz} % Required for drawing custom shapes
\usepackage[T1]{fontenc}
\usepackage[utf8]{inputenc} % Required for including letters with accents
\usepackage{enumitem} % Customize lists
\usepackage{longtable}
\usepackage{booktabs} % Required for nicer horizontal rules in tables
\usepackage{xcolor} % Required for specifying colors by name
\usepackage{listings} % listagens
\usepackage{float}
%----------------------------------------------------------------------------------
%	IMAGENS
%----------------------------------------------------------------------------------
\graphicspath{{Pictures/}} % Specifies the directory where pictures are stored
%----------------------------------------------------------------------------------
%	CAIXA DE LISTAGEM
%----------------------------------------------------------------------------------
% Definição para as caixas de listagens
\definecolor{codegray}{rgb}{0.5,0.5,0.5}
\definecolor{backcolour}{rgb}{0.95,0.95,0.92}

% Espaçamento dos Parágrafos
\setlength{\parindent}{0em}
\setlength{\parskip}{1em}

\lstset {
 aboveskip=3mm,
 framexleftmargin=2mm,
 xleftmargin=2mm,
 backgroundcolor=\color{backcolour},
 basicstyle={\small\ttfamily},
 belowskip=3mm,
 breaklines=true,
 breakatwhitespace=true,
 columns=flexible,
 commentstyle=\textit,
 frame=tb,
 keepspaces=true,
 keywordstyle=\color{blue}\bfseries,
 % language=Java, Python, HTML, CSS
 showstringspaces=false,
 showtabs=false,
 tabsize=3,
 literate=%
  {á}{{\'a}}1
  {é}{{\'e}}1
  {í}{{\'i}}1
  {ó}{{\'o}}1
  {ú}{{\'u}}1
  {â}{{\^a}}1
  {ê}{{\^e}}1
  {ã}{{\~a}}1
  {õ}{{\~o}}1
  {ç}{{\c{c}}}1
  {Á}{{\'A}}1
  {É}{{\'E}}1
  {Í}{{\'I}}1
  {Ó}{{\'O}}1
  {Ú}{{\'U}}1
  {Ê}{{\^E}}1
  {Ã}{{\~A}}1
  {Õ}{{\~O}}1
  {Ç}{{\c{C}}}1 
}
%----------------------------------------------------------------------------------
%	FONTS
%----------------------------------------------------------------------------------
\usepackage{avant} % Use the Avantgarde font for headings
\usepackage{mathptmx} % Use the Adobe Times Roman as the default text font together
\usepackage{microtype} % Slightly tweak font spacing for aesthetics
\usepackage[T1]{fontenc} % Use 8-bit encoding that has 256 glyphs
%----------------------------------------------------------------------------------
%	HYPERREF
%----------------------------------------------------------------------------------
\usepackage{hyperref}
\hypersetup{ pdfborder = {0 0 0}}
%----------------------------------------------------------------------------------
%	BIBLIOGRAPHY AND INDEX
%----------------------------------------------------------------------------------
% \usepackage[style=numeric,citestyle=numeric,sorting=nyt,sortcites=true,autopunct=true,hyperref=true,abbreviate=false,backref=true,backend=biber]{biblatex}
%\addbibresource{bibliography.bib} % BibTeX bibliography file
% \defbibheading{bibempty}{}
\usepackage{calc} % For simpler calculation - used for spacing the index letter headings correctly
\usepackage{makeidx} % Required to make an index
\makeindex % Tells LaTeX to create the files required for indexing
%----------------------------------------------------------------------------------
%	MAIN TABLE OF CONTENTS
%----------------------------------------------------------------------------------
\usepackage{titletoc} % Required for manipulating the table of contents
\setcounter{tocdepth}{1}
\contentsmargin{0cm} % Removes the default margin
% Part text styling
\titlecontents{part}[0cm]
{\addvspace{2pt}\centering\large\bfseries}
{}
{}
{}
%------------------------
% Chapter text styling
%------------------------
\titlecontents{chapter}[1.25cm] % Indentation
{\addvspace{12pt}\large\sffamily\bfseries} % Spacing and font options for chapters
{\color{blue!60}\contentslabel[\Large\thecontentslabel]{1.25cm}\color{blue!60}} % Chapter number
{\color{blue!60}}  
{} % No Page number
% {\color{blue!60}\normalsize\;\titlerule*[.5pc]{.}\;\thecontentspage} % Page number
%------------------------
% Section text styling
%------------------------
\titlecontents{section}[1.25cm] % Indentation
{\addvspace{3pt}\sffamily\bfseries} % Spacing and font options for sections
{\contentslabel[\thecontentslabel]{1.25cm}} % Section number
{}
{\color{black}\titlerule*[.5pc]{.}\;\thecontentspage} % Page number
[]
%------------------------
% Subsection text styling
%------------------------
\titlecontents{subsection}[1.25cm] % Indentation
{\addvspace{1pt}\sffamily\small} % Spacing and font options for subsections
{\contentslabel[\thecontentslabel]{1.25cm}} % Subsection number
{}
{\ \titlerule*[.5pc]{.}\;\thecontentspage} % Page number
[]
%------------------------
% List of figures
%------------------------
\titlecontents{figure}[0em]
{\addvspace{-5pt}\sffamily}
{\thecontentslabel\hspace*{1em}}
{}
{\ \titlerule*[.5pc]{.}\;\thecontentspage}
[]
%------------------------
% List of tables
%------------------------
\titlecontents{table}[0em]
{\addvspace{-5pt}\sffamily}
{\thecontentslabel\hspace*{1em}}
{}
{\ \titlerule*[.5pc]{.}\;\thecontentspage}
[]
%----------------------------------------------------------------------------------
%	MINI TABLE OF CONTENTS IN PART HEADS
%----------------------------------------------------------------------------------
% Chapter text styling
\titlecontents{lchapter}[0em] % Indenting
{\addvspace{25pt}\large\sffamily\bfseries} % Spacing and font options for chapters
{\color{blue}\contentslabel[\Large\thecontentslabel]{1.25cm}\color{blue}} % Chapter number
{}  
{\color{blue}\normalsize\sffamily\bfseries\;\titlerule*[.5pc]{.}\;\thecontentspage} % Page number
% Section text styling
\titlecontents{lsection}[0em] % Indenting
{\sffamily\small} % Spacing and font options for sections
{\contentslabel[\thecontentslabel]{1.25cm}} % Section number
{}
{}
% Subsection text styling
\titlecontents{lsubsection}[.5em] % Indentation
{\normalfont\footnotesize\sffamily} % Font settings
{}
{}
{}
%----------------------------------------------------------------------------------
%	PAGE HEADERS
%----------------------------------------------------------------------------------
\usepackage{fancyhdr} % Required for header and footer configuration
\pagestyle{fancy}
\renewcommand{\chaptermark}[1]{\markboth{\sffamily\normalsize\bfseries\chaptername\ \thechapter.\ #1}{}} % Chapter text font settings
\renewcommand{\sectionmark}[1]{\markright{\sffamily\normalsize\thesection\hspace{5pt}#1}{}} % Section text font settings
\fancyhf{} \fancyhead[LE,RO]{\sffamily\normalsize\thepage} % Font setting for the page number in the header
\fancyhead[LO]{\rightmark} % Print the nearest section name on the left side of odd pages
\fancyhead[RE]{\leftmark} % Print the current chapter name on the right side of even pages
\renewcommand{\headrulewidth}{0.5pt} % Width of the rule under the header
\addtolength{\headheight}{12pt} % Increase the spacing around the header slightly
\renewcommand{\footrulewidth}{0pt} % Removes the rule in the footer
\fancypagestyle{plain}{\fancyhead{}\renewcommand{\headrulewidth}{0pt}} % Style for when a plain pagestyle is specified
% Removes the header from odd empty pages at the end of chapters
\makeatletter
\renewcommand{\cleardoublepage}{
    \ifodd\c@page\else
    \hbox{}
    \vspace*{\fill}
    \thispagestyle{empty}
    \newpage
    \fi
}
%----------------------------------------------------------------------------------
%	THEOREM STYLES
%----------------------------------------------------------------------------------
\usepackage{amsmath,amsfonts,amssymb,amsthm} % For math equations, theorems, symbols, etc
\newcommand{\intoo}[2]{\mathopen{]}#1\,;#2\mathclose{[}}
\newcommand{\ud}{\mathop{\mathrm{{}d}}\mathopen{}}
\newcommand{\intff}[2]{\mathopen{[}#1\,;#2\mathclose{]}}
\newtheorem{notation}{Anotação}[chapter]
% Boxed/framed environments
\newtheoremstyle{bluenumbox}% % Theorem style name
{10pt}% Space above
{0pt}% Space below
{\normalfont}% % Body font
{}% Indent amount
{\small\bf\sffamily\color{blue}}% % Theorem head font
{\;}% Punctuation after theorem head
{0.25em}% Space after theorem head
{\small\sffamily\color{blue}\thmname{#1}\nobreakspace\thmnumber{\@ifnotempty{#1}{}\@upn{#2}}% Theorem text (e.g. Theorem 2.1)
\thmnote{\nobreakspace\the\thm@notefont\sffamily\bfseries\color{black}---\nobreakspace#3.}} % Optional theorem note
\renewcommand{\qedsymbol}{$\blacksquare$}% Optional qed square
% Boxed/framed environments
\newtheoremstyle{blacknumex}% Theorem style name
{5pt}% Space above
{5pt}% Space below
{\normalfont}% Body font
{} % Indent amount
{\small\bf\sffamily}% Theorem head font
{\;}% Punctuation after theorem head
{0.25em}% Space after theorem head
{\small\sffamily{\tiny\ensuremath{\blacksquare}}\nobreakspace\thmname{#1}\nobreakspace\thmnumber{\@ifnotempty{#1}{}\@upn{#2}}% Theorem text (e.g. Theorem 2.1)
\thmnote{\nobreakspace\the\thm@notefont\sffamily\bfseries---\nobreakspace#3.}}% Optional theorem note
% Boxed/framed environments
\newtheoremstyle{blacknumbox} % Theorem style name
{0pt}% Space above
{0pt}% Space below
{\normalfont}% Body font
{}% Indent amount
{\small\bf\sffamily}% Theorem head font
{\;}% Punctuation after theorem head
{0.25em}% Space after theorem head
{\small\sffamily\thmname{#1}\nobreakspace\thmnumber{\@ifnotempty{#1}{}\@upn{#2}}% Theorem text (e.g. Theorem 2.1)
\thmnote{\nobreakspace\the\thm@notefont\sffamily\bfseries---\nobreakspace#3.}}% Optional theorem note
% Non-boxed/non-framed environments
\newtheoremstyle{bluenum}% % Theorem style name
{5pt}% Space above
{5pt}% Space below
{\normalfont}% % Body font
{}% Indent amount
{\small\bf\sffamily\color{blue}}% % Theorem head font
{\;}% Punctuation after theorem head
{0.25em}% Space after theorem head
{\small\sffamily\color{blue}\thmname{#1}\nobreakspace\thmnumber{\@ifnotempty{#1}{}\@upn{#2}}% Theorem text (e.g. Theorem 2.1)
\thmnote{\nobreakspace\the\thm@notefont\sffamily\bfseries\color{black}---\nobreakspace#3.}} % Optional theorem note
\renewcommand{\qedsymbol}{$\blacksquare$}% Optional qed square
\makeatother
% Defines the theorem text style for each type of theorem to one of the three styles above
\newcounter{dummy} 
\numberwithin{dummy}{section}
\theoremstyle{bluenumbox}
\newtheorem{theoremeT}{Observação}
\newtheorem{problem}{Problema}[chapter]
\newtheorem{exerciseT}{Exercício}[chapter]
\theoremstyle{blacknumex}
\newtheorem{exampleT}{Exemplo}[chapter]
\theoremstyle{blacknumbox}
\newtheorem{vocabulary}{Vocabulário}[chapter]
\newtheorem{definitionT}{Definição}[section]
\newtheorem{dicaT}{Dica}
\theoremstyle{bluenum}
\newtheorem{proposition}[dummy]{Proposition}
%------------------------------------------------------------------------------------
%	DEFINITION OF COLORED BOXES
%------------------------------------------------------------------------------------
\RequirePackage[framemethod=default]{mdframed} % Required for creating the theorem, definition, exercise and dica boxes
% Theorem box
\newmdenv[skipabove=7pt,
    skipbelow=7pt,
    backgroundcolor=black!5,
    linecolor=blue,
    innerleftmargin=5pt,
    innerrightmargin=5pt,
    innertopmargin=5pt,
    leftmargin=0cm,
    rightmargin=0cm,
    innerbottommargin=5pt]{tBox}
% Exercise box	  
\newmdenv[skipabove=7pt,
    skipbelow=7pt,
    rightline=false,
    leftline=true,
    topline=false,
    bottomline=false,
    backgroundcolor=blue!10,
    linecolor=blue,
    innerleftmargin=5pt,
    innerrightmargin=5pt,
    innertopmargin=5pt,
    innerbottommargin=5pt,
    leftmargin=0cm,
    rightmargin=0cm,
    linewidth=4pt]{eBox}	
% Definition box
\newmdenv[skipabove=7pt,
    skipbelow=7pt,
    rightline=false,
    leftline=true,
    topline=false,
    bottomline=false,
    linecolor=blue,
    innerleftmargin=5pt,
    innerrightmargin=5pt,
    innertopmargin=0pt,
    leftmargin=0cm,
    rightmargin=0cm,
    linewidth=4pt,
    innerbottommargin=0pt]{dBox}	
% Caixa de Dica
\newmdenv[skipabove=0pt,
    skipbelow=7pt,
    rightline=false,
    leftline=true,
    topline=false,
    bottomline=false,
    linecolor=blue!60,
    backgroundcolor=black!5,
    innerleftmargin=5pt,
    innerrightmargin=5pt,
    innertopmargin=18pt,
    leftmargin=0cm,
    rightmargin=0cm,
    linewidth=4pt,
    innerbottommargin=5pt]{cBox}
% Creates an environment for each type of theorem and assigns it a theorem text style 
% from the Theorem Styles section above and a colored box from above
\newenvironment{theorem}{\begin{tBox}\begin{theoremeT}}{\end{theoremeT}\end{tBox}}
\newenvironment{exercise}{\begin{eBox}\begin{exerciseT}}{\hfill{\color{blue}\tiny\ensuremath{\blacksquare}}\end{exerciseT}\end{eBox}}
\newenvironment{definition}{\begin{dBox}\begin{definitionT}}{\end{definitionT}\end{dBox}}	
\newenvironment{example}{\begin{exampleT}}{\hfill{\tiny\ensuremath{\blacksquare}}\end{exampleT}}
\newenvironment{dica}{\begin{cBox}\begin{dicaT}}{\end{dicaT}\end{cBox}}
%------------------------------------------------------------------------------------
%	REMARK ENVIRONMENT
%------------------------------------------------------------------------------------
\newenvironment{remark}{\par\vspace{1pt}\small % Vertical white space above the remark and smaller font size
\begin{list}{}{
    \leftmargin=35pt % Indentation on the left
    \rightmargin=25pt}\item\ignorespaces % Indentation on the right
\makebox[-2.5pt]{\begin{tikzpicture}[overlay]
    \node[draw=blue!60,line width=1pt,circle,fill=blue!25,font=\sffamily\bfseries,inner sep=2pt,outer sep=0pt] at (-15pt,0pt){\textcolor{blue}{F}};\end{tikzpicture}} % Orange R in a circle
    \advance\baselineskip -1pt}{\end{list}\vskip5pt} % Tighter line spacing and white space after remark
%------------------------------------------------------------------------------------
%	SECTION NUMBERING IN THE MARGIN
%------------------------------------------------------------------------------------
\makeatletter
\renewcommand{\@seccntformat}[1]{\llap{\textcolor{blue}{\csname the#1\endcsname}\hspace{1em}}}                    
\renewcommand{\section}{\@startsection{section}{1}{\z@}
    {-4ex \@plus -1ex \@minus -.4ex}
    {1ex \@plus.2ex }
    {\normalfont\large\sffamily\bfseries}}
\renewcommand{\subsection}{\@startsection {subsection}{2}{\z@}
    {-3ex \@plus -0.1ex \@minus -.4ex}
    {0.5ex \@plus.2ex }
    {\normalfont\sffamily\bfseries}}
\renewcommand{\subsubsection}{\@startsection {subsubsection}{3}{\z@}
    {-2ex \@plus -0.1ex \@minus -.2ex}
    {.2ex \@plus.2ex }
    {\normalfont\small\sffamily\bfseries}} 
\renewcommand\paragraph{\@startsection{paragraph}{4}{\z@}
    {-2ex \@plus-.2ex \@minus .2ex}
    {.1ex}
    {\normalfont\small\sffamily\bfseries}}
%------------------------------------------------------------------------------------
%	PART HEADINGS
%------------------------------------------------------------------------------------
% numbered part in the table of contents
\newcommand{\@mypartnumtocformat}[2]{
    \setlength\fboxsep{0pt}
    \noindent\colorbox{blue!20}{\strut\parbox[c][.7cm]{\ecart}{\color{blue!70}\Large\sffamily\bfseries\centering#1}}\hskip\esp\colorbox{blue!40}{\strut\parbox[c][.7cm]{\linewidth-\ecart-\esp}{\Large\sffamily\centering#2}}}
% unnumbered part in the table of contents
\newcommand{\@myparttocformat}[1]{
    \setlength\fboxsep{0pt}
    \noindent\colorbox{blue!40}{\strut\parbox[c][.7cm]{\linewidth}{\Large\sffamily\centering#1}}}
\newlength\esp
\setlength\esp{4pt}
\newlength\ecart
\setlength\ecart{1.2cm-\esp}
\newcommand{\thepartimage}{}
\newcommand{\partimage}[1]{\renewcommand{\thepartimage}{#1}}
\def\@part[#1]#2{
    \ifnum \c@secnumdepth >-2\relax
        \refstepcounter{part}
        \addcontentsline{toc}{part}{\texorpdfstring{\protect\@mypartnumtocformat{\thepart}{#1}}{\partname~\thepart\ ---\ #1}}
    \else
        \addcontentsline{toc}{part}{\texorpdfstring{\protect\@myparttocformat{#1}}{#1}}
    \fi
    \startcontents
    \markboth{}{}
    {\thispagestyle{empty}
    \begin{tikzpicture}[remember picture,overlay]
    \node at (current page.north west){\begin{tikzpicture}[remember picture,overlay]
    \fill[blue!20](0cm,0cm) rectangle (\paperwidth,-\paperheight);
    \node[anchor=north] at (4cm,-3.25cm){\color{blue!40}\fontsize{220}{100}\sffamily\bfseries\thepart}; 
    \node[anchor=south east] at (\paperwidth-1cm,-\paperheight+1cm){\parbox[t][][t]{8.5cm}{
    \printcontents{l}{0}{\setcounter{tocdepth}{1}}
    }};
    \node[anchor=north east] at (\paperwidth-1.5cm,-3.25cm){\parbox[t][][t]{15cm}{\strut\raggedleft\color{white}\fontsize{30}{30}\sffamily\bfseries#2}};
    \end{tikzpicture}};
    \end{tikzpicture}}
    \@endpart}
\def\@spart#1{
\startcontents
\phantomsection
{\thispagestyle{empty}
\begin{tikzpicture}[remember picture,overlay]
\node at (current page.north west){\begin{tikzpicture}[remember picture,overlay]
    \fill[blue!20](0cm,0cm) rectangle (\paperwidth,-\paperheight);
    \node[anchor=north east] at (\paperwidth-1.5cm,-3.25cm){\parbox[t][][t]{15cm}{\strut\raggedleft\color{white}\fontsize{30}{30}\sffamily\bfseries#1}};
    \end{tikzpicture}};
\end{tikzpicture}}
\addcontentsline{toc}{part}{\texorpdfstring{
    \setlength\fboxsep{0pt}
    \noindent\protect\colorbox{blue!40}{\strut\protect\parbox[c][.7cm]{\linewidth}{\Large\sffamily\protect\centering #1\quad\mbox{}}}}{#1}}
    \@endpart}
\def\@endpart{\vfil\newpage
    \if@twoside
        \if@openright
            \null
            \thispagestyle{empty}
            \newpage
        \fi
    \fi
    \if@tempswa
    \twocolumn
    \fi}
%------------------------------------------------------------------------------------
%	CHAPTER HEADINGS
%------------------------------------------------------------------------------------
% A switch to conditionally include a picture, implemented by  Christian Hupfer
\newif\ifusechapterimage
\usechapterimagetrue
\newcommand{\thechapterimage}{}
\newcommand{\chapterimage}[1]{\ifusechapterimage\renewcommand{\thechapterimage}{#1}\fi}
\newcommand{\autodot}{.}
\def\@makechapterhead#1{
    {\parindent \z@ \raggedright \normalfont
    \ifnum \c@secnumdepth >\m@ne
    \if@mainmatter
    \begin{tikzpicture}[remember picture,overlay]
    \node at (current page.north west)
    {\begin{tikzpicture}[remember picture,overlay]
    \node[anchor=north west,inner sep=0pt] at (0,0) {\ifusechapterimage\includegraphics[width=\paperwidth]{\thechapterimage}\fi};
    \draw[anchor=west] (\Gm@lmargin,-9cm) node [line width=2pt,rounded corners=15pt,draw=blue!60,fill=white,fill opacity=0.5,inner sep=15pt]{\strut\makebox[22cm]{}};
    \draw[anchor=west] (\Gm@lmargin+.3cm,-9cm) node {\huge\sffamily\bfseries\color{black}\thechapter\autodot~#1\strut};
    \end{tikzpicture}};
    \end{tikzpicture}
    \else
    \begin{tikzpicture}[remember picture,overlay]
    \node at (current page.north west)
    {\begin{tikzpicture}[remember picture,overlay]
    \node[anchor=north west,inner sep=0pt] at (0,0) {\ifusechapterimage\includegraphics[width=\paperwidth]{\thechapterimage}\fi};
    \draw[anchor=west] (\Gm@lmargin,-9cm) node [line width=2pt,rounded corners=15pt,draw=blue,fill=white,fill opacity=0.5,inner sep=15pt]{\strut\makebox[22cm]{}};
    \draw[anchor=west] (\Gm@lmargin+.3cm,-9cm) node {\huge\sffamily\bfseries\color{black}#1\strut};
    \end{tikzpicture}};
    \end{tikzpicture}
    \fi\fi\par\vspace*{180\p@}}}
\def\@makeschapterhead#1{
\begin{tikzpicture}[remember picture,overlay]
    \node at (current page.north west)
    {\begin{tikzpicture}[remember picture,overlay]
        \node[anchor=north west,inner sep=0pt] at (0,0) {\ifusechapterimage\includegraphics[width=\paperwidth]{\thechapterimage}\fi};
        \draw[anchor=west] (\Gm@lmargin,-9cm) node [line width=2pt,rounded corners=15pt,draw=blue!60,fill=white,fill opacity=0.5,inner sep=15pt]{\strut\makebox[22cm]{}};
        \draw[anchor=west] (\Gm@lmargin+.3cm,-9cm) node {\huge\sffamily\bfseries\color{black}#1\strut};
    \end{tikzpicture}};
\end{tikzpicture}
\par\vspace*{180\p@}}
\makeatother
%------------------------------------------------------------------------------------
%	HYPERLINKS IN THE DOCUMENTS
%------------------------------------------------------------------------------------
\usepackage{url}
\usepackage{bookmark}
\bookmarksetup{
    open,
    numbered,
    addtohook={
        \ifnum\bookmarkget{level}=0 % chapter
        \bookmarksetup{bold}%
        \fi
        \ifnum\bookmarkget{level}=-1 % part
        \bookmarksetup{color=blue,bold}
        \fi
    }
}
\raggedbottom
%------------------------------------------------------------------------------------
%	TABLES - TABULAR
%------------------------------------------------------------------------------------
\setlength{\arrayrulewidth}{0.5mm}
% \setlength{\tabcolsep}{18pt}
\renewcommand{\arraystretch}{1.3}




%----------------------------------------------------------------------------------------
% INICIAL
%----------------------------------------------------------------------------------------
\title{	
 \normalfont\normalsize
 \textsc{Centro Universitário IESB}\\
 \vspace{5pt} % Whitespace
 \rule{\linewidth}{0.5pt}\\
 \vspace{20pt} % Whitespace
 {\huge Notas sobre Flutter}\\
 \vspace{12pt} % Whitespace
 \rule{\linewidth}{2pt}
}

\author{\LARGE Fernando Anselmo}
\date{v.1.0 em \normalsize\today}

\begin{document}

\maketitle

Não contém neste programas completos ou descrições, está no estilo de \textit{HANDS-ON} (mão na massa) composto por anotações soltas para auxiliar no desenvolvimento ou problemas que podem surgir.

%----------------------------------------------------------------------------------------
% TEXTO
%----------------------------------------------------------------------------------------
\section{Básico}

Ambiente Dart para testes: \\
\url{https://dartpad.dev/}

Novo projeto: \\
{\ttfamily\$ flutter create ---org dev.decus -a java meu-projeto}

Reconstruir um projeto: Na pasta do projeto
\\
{\ttfamily\$ flutter create .}

Parar AVD:
\\
{\ttfamily\$ adb shell
\\
sync \&\& reboot -p}

Rodar:
\\
{\ttfamily\$ flutter run [---profile] [---release]}

%------------------------------------------------
\subsection{Problemas que podem acontecer}

Não encontrou \textbf{material.dart}: \\
{\ttfamily\$ flutter doctor -v} \\
{\ttfamily\$ flutter packages get} \\
{\ttfamily\$ flutter clean}

Resolução Geral:
\\
{\ttfamily\$ flutter channel dev} \\
{\ttfamily\$ flutter doctor} \\
{\ttfamily\$ flutter channel master} \\
{\ttfamily\$ flutter doctor}

Definir caminhos das variáveis: (os caminhos se referem ao meu SO) \\
{\ttfamily\$ flutter config ---android-sdk=``/home/fernando/Android/Sdk''} \\
{\ttfamily\$ flutter config ---android-studio-dir=``/opt/android-studio''} \\
{\ttfamily\$ /home/fernando/Android/Sdk/tools/bin/sdkmanager ---install ``cmdline-tools;latest''} \\
{\ttfamily\$ flutter doctor ---android-licenses}

Habilitar ambientes: \\
{\ttfamily\$ flutter config ---enable-linux-desktop} \\
{\ttfamily\$ flutter config ---enable-windows-desktop} \\
{\ttfamily\$ flutter config ---enable-macos-desktop}

Não use o comando \textbf{print}: \\
{\ttfamily log(texto);} e adicionar o pacote: {\ttfamily import 'dart:developer';}

Para implementar classe com a anotação @immutable: \\
Todas variáveis devem ser \textbf{final} e o construtor \textbf{const}.

No caso de Cores: \\
Ao invés de: {\ttfamily Colors.grey[300]} usar {\ttfamily Color(0xFFE0E0E0)} \\
Ao invés de: {\ttfamily Colors.grey[100]} usar {\ttfamily Color(0xFFF5F5F5)}

Para resolver o \textbf{accentColor} depreciado:

1º Criar um objeto de ThemeData:
\begin{lstlisting}[]
 final ThemeData theme = ThemeData(
  primarySwatch: white,
 );
\end{lstlisting}

2º Usar este objeto no MaterialApp:
\begin{lstlisting}[]
 theme: theme.copyWith(
  colorScheme: theme.colorScheme.copyWith(secondary: Colors.black),
 ),
\end{lstlisting}

Tema \textbf{Dark}:
\begin{lstlisting}[]
 MaterialApp(
  themeMode: ThemeMode.dark,
  theme: _lightTheme,
  darkTheme: _darkTheme,
  ...
 ),	
\end{lstlisting}

Aparecer além da barra:
\begin{lstlisting}[]
 return Scaffold(
  extendBodyBehindAppBar: true,
 ); 
\end{lstlisting}

Mudar a barra superior e inferior do Telefone:
\begin{lstlisting}[]
 void main() {
  var systemUiOverlayStyle = const SystemUiOverlayStyle(
   statusBarColor: Colors.orangeAccent, // ou Colors.transparent,
   systemNavigationBarColor: Colors.orangeAccent, // ou Colors.transparent,
  );
  SystemChrome.setSystemUIOverlayStyle(
   systemUiOverlayStyle,
  );
  runApp(const MyApp());
 }
\end{lstlisting}

\textbf{FlatButton} depreciado, trocar para \textbf{TextButton} com estilo:
\begin{lstlisting}[]
 final ButtonStyle flatButtonStyle = TextButton.styleFrom(
  backgroundColor: Colors.grey,
  padding: const EdgeInsets.all(0),
 );

 ... 
   TextButton(
    style: flatButtonStyle,
   ...
\end{lstlisting}

Erro no http: \\
Criar uma: {\ttfamily var url = Uri.parse(END\_URL);} \\
E usar esta no HTTP: {\ttfamily final response = await http.get(url);}

Ou mesmo usar uma url mais completa:
\begin{lstlisting}[]
 Uri apiUri = Uri.https('api.tvmaze.com', 'singlesearch/shows',
  {'q': 'house', 'embed': 'episodes'});
\end{lstlisting}

Definir um elemento completo de Cor:
\begin{lstlisting}[]
const MaterialColor white = MaterialColor(
  0xFFFFFFFF,
  <int, Color>{
    50: Color(0xFFFFFFFF),
    100: Color(0xFFFFFFFF),
    200: Color(0xFFFFFFFF),
    300: Color(0xFFFFFFFF),
    400: Color(0xFFFFFFFF),
    500: Color(0xFFFFFFFF),
    600: Color(0xFFFFFFFF),
    700: Color(0xFFFFFFFF),
    800: Color(0xFFFFFFFF),
    900: Color(0xFFFFFFFF),
  },
);
\end{lstlisting}

%----------------------------------------------------------------------------------------
\section{Widgets}

%----------------------------------------------------------------------------------------
\subsection{Contâineres}
Com \textit{children}:
\begin{itemize}[nolistsep]
	\item {\ttfamily Column(children: [])}
	\item {\ttfamily Row(children: [])}
	\item {\ttfamily Wrap(children: [])} - Quebra a linha
	\item {\ttfamily Stack(children: [\_image, \_text,],)} - Imagem mesclada com texto
\end{itemize}

Com \textit{child}:
\begin{itemize}[nolistsep]
	\item {\ttfamily Container(child: )}
	\item {\ttfamily Card(child: )}
	\item {\ttfamily Align(alignment: Alignment.[tipo], child: )}
	\item {\ttfamily Padding(padding: const EdgeInsets.all(5.0), child: )}
	\item {\ttfamily SafeArea(child: )}
	\item {\ttfamily ClipRRect(borderRadius: BorderRadius.circular(25), child: )}
	\item {\ttfamily Center(child: )} - Centraliza o texto
	\item {\ttfamily FittedBox(child: )} - Maximiza o texto
	\item {\ttfamily Visibility(visible: true/false, child: )} - faz o objeto aparecer ou desaparecer
	\item {\ttfamily Positioned(top: , left: , height: , width: , child: )}
\end{itemize}

Decorar um \textit{Container}:
\begin{lstlisting}[]
 Container(
  decoration: BoxDecoration(
   borderRadius: BorderRadius.circular(8.0),
   color: Colors.blueAccent,
  ),
 ),
\end{lstlisting}

%----------------------------------------------------------------------------------------
\subsection{Divisores}
Obter um determinado espaçamento:
\begin{itemize}[nolistsep]
	\item {\ttfamily Expanded(flex: [fator], child: )}
	\item {\ttfamily Flexible(flex: [fator], child: )}
	\item {\ttfamily SizedBox(height: alt, width: larg, child: )}
	\item {\ttfamily Placeholder(fallbackHeight: 200, color: Colors.blue, strokeWidth: 5)}
\end{itemize}

%----------------------------------------------------------------------------------------
\subsection{Relativos a Imagem}
Imagens podem vir de Asset ou Web: \\
Asset: {\ttfamily Image.asset('assets/images/notfound.png')} \\
Web: {\ttfamily Image.network('endereço', fit: BoxFit.cover)}

Não quebrar uma imagem vinda da Web:
\begin{lstlisting}[]
 Hero(
  tag: 'hero', 
  child: Image.network('https://food.bolt.eu/og-img.jpg'),
 ),
\end{lstlisting}

Criar uma imagem circular:
\begin{lstlisting}[]
 CircleAvatar(
  radius: ,
  backgroundColor: ,
  child: _image,
 )
\end{lstlisting}

Exata proporção de uma imagem:
\begin{lstlisting}[]
 AspectRatio(
  aspectRatio: 1 / 2, (ou 3/4, 7/2)
  child: _image,
 ),
\end{lstlisting}

Mudar a opacidade de uma imagem:
\begin{lstlisting}[]
 Opacity(
  child: _image,
  opacity: 0.3,
 ),	
\end{lstlisting}

Adicionar um filtro de cor:
\begin{lstlisting}[]
 ColorFiltered(
  colorFilter: const ColorFilter.mode(
   Colors.red,
   BlendMode.modulate,
  ),
  child: _image,
 ),	
\end{lstlisting}

Efeito de aparecer (fade) em imagens da Internet:
\begin{lstlisting}[]
 FadeInImage.assetNetwork(
  placeholder: 'images/loading.gif',
  fadeOutDuration: Duration(seconds: 2),
  image: 'https://picsum.photos/250?image=9',
 ),	
\end{lstlisting}

%----------------------------------------------------------------------------------------
\subsection{Relativos a Textos (Labels)}
Texto padrão com estilo:
\begin{lstlisting}[]
 Text(
  episodio['name'] ?? '',
  style: TextStyle(color: corEpisodio, fontSize: 20),
 ),
\end{lstlisting}

Texto selecionável:
\begin{lstlisting}[]
 SelectableText('', style: [estilo])
\end{lstlisting}

Texto com elementos de estilo interno:
\begin{lstlisting}[]
 child: RichText(
  text: TextSpan(
   style: TextStyle(fontSize: 20, color: Colors.black),
   children: <TextSpan>[
    const TextSpan(text: 'Este é um exemplo de '),
    TextSpan(
     text: 'RichText',
     style: TextStyle(
      fontWeight: FontWeight.bold,
      color: Colors.blue[400],
     ),
    ),
   ],
  ),
 ),
\end{lstlisting}

%----------------------------------------------------------------------------------------
\subsection{Relativos as Listas de Widgets}
Barra de rolagem:
\begin{lstlisting}[]
 Scrollbar(
  isAlwaysShown: true,
  child: ...
 ), 
\end{lstlisting}

Slider - Intervalo de Valores: 
\begin{lstlisting}[]
 Slider(
  value: valorCorrente,
  onChanged: (novoValor) {
   setState(() {
    valorCorrente = novoValor; 
   });
  },
  min: 0,
  max: 100,
 ),
\end{lstlisting}

Chip - Lista de algo
\begin{lstlisting}[]
 Chip(
  avatar: CircleAvatar(
   child: Text(emails[index].substring(0, 1)),
  ),
  label: Text(emails[index]),
  onDeleted: () {
   setState(() {
    emails.removeAt(index);
   });
  },
 ),	
\end{lstlisting}

PageView - Várias Visões:
\begin{lstlisting}[]
 PageView(
  scrollDirection: Axis.vertical,
  children: _widgets,
 ),	
\end{lstlisting}

Tabela:
\begin{lstlisting}[]
 DataTable(
  columns: [
   DataColumns(label: nomcol1),
   DataColumns(label: nomcol2),
  ],
  rows: [
   DataRows(cells: [
    DataCell(Text(val1c1)),
    DataCell(Text(val1c2)),
   ]),
   DataRows(cells: [
    DataCell(Text(val2c1)),
    DataCell(Text(val2c2)),
   ]),
  ],
 )
\end{lstlisting}

Lista lateral:
\begin{lstlisting}[]
 return Scaffold(
  ...
  drawer: Drawer(
   child: _listView,
  ),
 );
\end{lstlisting}

%----------------------------------------------------------------------------------------
\subsection{Diversos e úteis}

Dica rápida:
\begin{lstlisting}[]
 Tooltip(
  message: 'Aqui a dica', 
  child: [widget]
 ),
\end{lstlisting}

Future.delayed - Espera um tempo:
\begin{lstlisting}[]
 onPressed: () async {
  await Future.delayed(const Duration(seconds: 1));
  ...
 }
\end{lstlisting}

FutureBuilder - Widget:

1º) Construir o método que vai dar a resposta:
\begin{lstlisting}[]
Future<String> getData() async {
  await Future.delayed(const Duration(seconds: 1));
  // throw "Para testar o erro";
  return "Funciona...";
}
\end{lstlisting}

2º) Ação no Scaffold:
\begin{lstlisting}[]
 return Scaffold(
   ...
   body: Center(
    child: FutureBuilder(
     future: getData(),
     builder: (context, snapshot) {
      if (snapshot.connectionState == ConnectionState.waiting) {
       return const CircularProgressIndicator.adaptive();
      }
      if (snapshot.hasError) {
       return Text(snapshot.error);
      } else {
       return Column(
        mainAxisSize: MainAxisSize.min,
        children: [
         Text(snapshot.data.toString()),
         ElevatedButton(
          onPressed: () {
           setState(() {});
          },
          child: const Text("Refresh"),
         ),
        ],
       );
      }
     }
    ),
   ),
 );
\end{lstlisting}

Quando tiver uma espera:
\begin{lstlisting}[]
 import 'dart:io'; // import
 import 'package:flutter/cupertino.dart'; // Cupertino

 Platform.isAndroid
  ? CircularProgressIndicator(): CupertinoActivityIndicator()
\end{lstlisting}

Mensagem SnackBar:
\begin{lstlisting}[]
 child: Builder(
  builder: (context) => GestureDetector(
   onTap: () {
    ScaffoldMessenger.of(context).showSnackBar(
     const SnackBar(
      duration: Duration(seconds: 1),
      content: Text('This is the Snackbar'),
     ),
    );
   },
  ),
 ),
\end{lstlisting}

Criar um QrCode:
\begin{lstlisting}[]
 QrImage(
  data: 'o que quiser aqui',
  version: QrVersions.auto,
  size: 200.0,
 ),
\end{lstlisting}


%----------------------------------------------------------------------------------------
\section{Top Pacotes e Plugins}

Página dos Pacotes:
\\
\url{https://pub.dev/}

\url{https://pub.dev/packages/introduction_screen} \\
{\ttfamily\$ flutter pub add introduction\_screen}

\url{https://pub.dev/packages/flutter_native_splash} \\
{\ttfamily\$ flutter pub add flutter\_native\_splash}

\url{https://pub.dev/packages/google_fonts} \\
{\ttfamily\$ flutter pub add google\_fonts}

Ex: 
\begin{lstlisting}[]
 ttfamilyText('Algo', style: GoogleFonts.aguafinaScript().copyWith(fontSize: 40),),
\end{lstlisting}

\url{https://pub.dev/packages/flutter_launch_icons} \\
{\ttfamily\$ flutter pub add flutter\_launch\_icons}

\url{https://pub.dev/packages/fluttertoast} \\
{\ttfamily\$ flutter pub add fluttertoast} 

Ex:
\begin{lstlisting}[]
 onPressed: () {
  Fluttertoast.showToast(
   msg: 'Mensagem da Torrada',
   backgroundColor: Colors.deepOrange,
   textColor: Colors.white,
   fontSize: 16.0,
  );
 },
\end{lstlisting}

\url{https://pub.dev/packages/screenshot} \\
{\ttfamily\$ flutter pub add screenshot}

\subsection{Plugins para o VS Code}
\begin{itemize}[nolistsep]
	\item Dart - Dart Code
	\item Flutter - Dart Code
	\item Material Icon Theme - Phillipp Kief
	\item Pubspec Assist - Jeroen Meijer
	\item Android Emulator Launcher - Dannark
	\item Awesome Flutter Snippets - Neevash Ramdial
	\item Error Lens - Alexander
\end{itemize}

%----------------------------------------------------------------------------------------
\section{Dicas de Dart}

Ao invés de usar: {\ttfamily episodio['name'] == null ? '' : episodio['name']} \\
Converter para: {\ttfamily episodio['name'] ?? ''}

Obter o tamanho da tela:
\begin{lstlisting}[]
 Size size = MediaQuery.of(context).size;
 double height = size.height;
 double width = size.width;
\end{lstlisting}

Ao invés de usar: {\ttfamily String text = '$\setminus$\$100';} \\
Converter para: {\ttfamily String text = r'\$100';}

\end{document}