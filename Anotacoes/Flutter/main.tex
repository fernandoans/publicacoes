%----------------------------------------------------------------------------------------
% PACOTES E OUTRAS CONFIGURAÇÕES
%----------------------------------------------------------------------------------------
\documentclass[11pt]{scrartcl}
%----------------------------------------------------------------------------------------
%	PACKAGES AND OTHER DOCUMENT CONFIGURATIONS
%----------------------------------------------------------------------------------------

\usepackage{amsmath, amsfonts, amsthm} % Math packages
\usepackage{listings} % Code listings, with syntax highlighting
\usepackage[brazil]{babel} % padronizar a linguagem
\usepackage[utf8]{inputenc} % permitir a acentuação
\usepackage{graphicx} % Required for inserting images
\graphicspath{{Figures/}{./}} % Specifies where to look for included images (trailing slash required)
\usepackage{booktabs} % Required for better horizontal rules in tables
\numberwithin{equation}{section} % Number equations within sections (i.e. 1.1, 1.2, 2.1, 2.2 instead of 1, 2, 3, 4)
\numberwithin{figure}{section} % Number figures within sections (i.e. 1.1, 1.2, 2.1, 2.2 instead of 1, 2, 3, 4)
\numberwithin{table}{section} % Number tables within sections (i.e. 1.1, 1.2, 2.1, 2.2 instead of 1, 2, 3, 4)
\setlength\parindent{0pt} % Removes all indentation from paragraphs
\usepackage{enumitem} % Required for list customisation
\setlist{noitemsep} % No spacing between list items
\usepackage{color}    % definir cores
\usepackage{listings} % listagens
\usepackage{url} % codigo para as URLs

%----------------------------------------------------------------------------------------
%	DOCUMENT MARGINS
%----------------------------------------------------------------------------------------

% Espaçamento dos Parágrafos
\setlength{\parindent}{0em}
\setlength{\parskip}{1em}

\usepackage{geometry} % Required for adjusting page dimensions and margins

\geometry{
	paper=a4paper, % Paper size, change to letterpaper for US letter size
	top=2.5cm, % Top margin
	bottom=3cm, % Bottom margin
	left=2cm, % Left margin
	right=2cm, % Right margin
	headheight=0.5cm, % Header height
	footskip=1.5cm, % Space from the bottom margin to the baseline of the footer
	headsep=0.75cm, % Space from the top margin to the baseline of the header
	%showframe, % Uncomment to show how the type block is set on the page
}

%----------------------------------------------------------------------------------------
%	FONTS
%----------------------------------------------------------------------------------------

\usepackage[utf8]{inputenc} % Required for inputting international characters
\usepackage[T1]{fontenc} % Use 8-bit encoding
\usepackage{fourier} % Use the Adobe Utopia font for the document

%----------------------------------------------------------------------------------------
%	SECTION TITLES
%----------------------------------------------------------------------------------------
\usepackage{sectsty} % Allows customising section commands
\sectionfont{\vspace{6pt}\centering\normalfont\scshape} % \section{} styling
\subsectionfont{\normalfont\bfseries} % \subsection{} styling
\subsubsectionfont{\normalfont\itshape} % \subsubsection{} styling
\paragraphfont{\normalfont\scshape} % \paragraph{} styling

%----------------------------------------------------------------------------------------
%	HEADERS AND FOOTERS
%----------------------------------------------------------------------------------------
\usepackage{scrlayer-scrpage} % Required for customising headers and footers

\ohead*{} % Right header
\ihead*{} % Left header
\chead*{} % Centre header

\ofoot*{} % Right footer
\ifoot*{} % Left footer
\cfoot*{\pagemark} % Centre footer

%-----------------------------------------------------------------------------
% Definição para as caixas de listagens
%-----------------------------------------------------------------------------
\definecolor{codegray}{rgb}{0.5,0.5,0.5}
\definecolor{backcolour}{rgb}{0.95,0.95,0.92}
\lstset {
	aboveskip=3mm,
	backgroundcolor=\color{backcolour},
	basicstyle={\small\ttfamily},
	belowskip=3mm,
	breaklines=true,
	breakatwhitespace=true,
	columns=flexible,
	commentstyle=\textit,
	extendedchars=true,
	frame=tb,
	keepspaces=true,
	keywordstyle=\color{blue}\bfseries,
	language=HTML,
	numbers=left,
	numbersep=5pt,
	numberstyle=\tiny\color{codegray},
	showstringspaces=false,
	showtabs=false,
	tabsize=3,
	literate=%
	{á}{{\'a}}1
	{é}{{\'e}}1
	{í}{{\'i}}1
	{ó}{{\'o}}1
	{ú}{{\'u}}1
	{â}{{\^a}}1
	{ê}{{\^e}}1
	{ã}{{\~a}}1
	{õ}{{\~o}}1
	{ç}{{\c{c}}}1
	{Á}{{\'A}}1
	{É}{{\'E}}1
	{Í}{{\'I}}1
	{Ó}{{\'O}}1
	{Ú}{{\'U}}1
	{Ê}{{\^E}}1
	{Ã}{{\~A}}1
	{Õ}{{\~O}}1
	{Ç}{{\c{C}}}1
}


%----------------------------------------------------------------------------------------
% INICIAL
%----------------------------------------------------------------------------------------
\title{	
 \normalfont\normalsize
 \textsc{Centro Universitário IESB}\\
 \vspace{5pt} % Whitespace
 \rule{\linewidth}{0.5pt}\\
 \vspace{20pt} % Whitespace
 {\huge Notas sobre Flutter}\\
 \vspace{12pt} % Whitespace
 \rule{\linewidth}{2pt}
}

\author{\LARGE Fernando Anselmo}
\date{v.1.0 em \normalsize\today}

\begin{document}

\maketitle

Não contém neste programas completos ou descrições, está no estilo de \textit{HANDS-ON} (mão na massa) composto por anotações soltas para auxiliar no desenvolvimento ou problemas que podem surgir.

%----------------------------------------------------------------------------------------
% TEXTO
%----------------------------------------------------------------------------------------
\section{Básico}

Ambiente Dart para testes: \\
\url{https://dartpad.dev/}

Novo projeto: \\
{\ttfamily\$ flutter create ---org dev.decus -a java meu-projeto}

Reconstruir um projeto: Na pasta do projeto
\\
{\ttfamily\$ flutter create .}

Parar AVD:
\\
{\ttfamily\$ adb shell
\\
sync \&\& reboot -p}

Rodar:
\\
{\ttfamily\$ flutter run [---profile] [---release]}

%------------------------------------------------
\subsection{Problemas que podem acontecer}

Não encontrou \textbf{material.dart}: \\
{\ttfamily\$ flutter doctor -v} \\
{\ttfamily\$ flutter packages get} \\
{\ttfamily\$ flutter clean}

Resolução Geral:
\\
{\ttfamily\$ flutter channel dev} \\
{\ttfamily\$ flutter doctor} \\
{\ttfamily\$ flutter channel master} \\
{\ttfamily\$ flutter doctor}

Definir caminhos das variáveis: (os caminhos se referem ao meu SO) \\
{\ttfamily\$ flutter config ---android-sdk=``/home/fernando/Android/Sdk''} \\
{\ttfamily\$ flutter config ---android-studio-dir=``/opt/android-studio''} \\
{\ttfamily\$ /home/fernando/Android/Sdk/tools/bin/sdkmanager ---install ``cmdline-tools;latest''} \\
{\ttfamily\$ flutter doctor ---android-licenses}

Habilitar ambientes: \\
{\ttfamily\$ flutter config ---enable-linux-desktop} \\
{\ttfamily\$ flutter config ---enable-windows-desktop} \\
{\ttfamily\$ flutter config ---enable-macos-desktop}

Não use o comando \textbf{print}: \\
{\ttfamily log(texto);} e adicionar o pacote: {\ttfamily import 'dart:developer';}

Para implementar classe com a anotação @immutable: \\
Todas variáveis devem ser \textbf{final} e o construtor \textbf{const}.

No caso de Cores: \\
Ao invés de: {\ttfamily Colors.grey[300]} usar {\ttfamily Color(0xFFE0E0E0)} \\
Ao invés de: {\ttfamily Colors.grey[100]} usar {\ttfamily Color(0xFFF5F5F5)}

Para resolver o \textbf{accentColor} depreciado:

1º Criar um objeto de ThemeData:
\begin{lstlisting}[]
 final ThemeData theme = ThemeData(
  primarySwatch: white,
 );
\end{lstlisting}

2º Usar este objeto no MaterialApp:
\begin{lstlisting}[]
 theme: theme.copyWith(
  colorScheme: theme.colorScheme.copyWith(secondary: Colors.black),
 ),
\end{lstlisting}

Tema \textbf{Dark}:
\begin{lstlisting}[]
 MaterialApp(
  themeMode: ThemeMode.dark,
  theme: _lightTheme,
  darkTheme: _darkTheme,
  ...
 ),	
\end{lstlisting}

Aparecer além da barra:
\begin{lstlisting}[]
 return Scaffold(
  extendBodyBehindAppBar: true,
 ); 
\end{lstlisting}

Mudar a barra superior e inferior do Telefone:
\begin{lstlisting}[]
 void main() {
  var systemUiOverlayStyle = const SystemUiOverlayStyle(
   statusBarColor: Colors.orangeAccent, // ou Colors.transparent,
   systemNavigationBarColor: Colors.orangeAccent, // ou Colors.transparent,
  );
  SystemChrome.setSystemUIOverlayStyle(
   systemUiOverlayStyle,
  );
  runApp(const MyApp());
 }
\end{lstlisting}

\textbf{FlatButton} depreciado, trocar para \textbf{TextButton} com estilo:
\begin{lstlisting}[]
 final ButtonStyle flatButtonStyle = TextButton.styleFrom(
  backgroundColor: Colors.grey,
  padding: const EdgeInsets.all(0),
 );

 ... 
   TextButton(
    style: flatButtonStyle,
   ...
\end{lstlisting}

Erro no http: \\
Criar uma: {\ttfamily var url = Uri.parse(END\_URL);} \\
E usar esta no HTTP: {\ttfamily final response = await http.get(url);}

Ou mesmo usar uma url mais completa:
\begin{lstlisting}[]
 Uri apiUri = Uri.https('api.tvmaze.com', 'singlesearch/shows',
  {'q': 'house', 'embed': 'episodes'});
\end{lstlisting}

Definir um elemento completo de Cor:
\begin{lstlisting}[]
const MaterialColor white = MaterialColor(
  0xFFFFFFFF,
  <int, Color>{
    50: Color(0xFFFFFFFF),
    100: Color(0xFFFFFFFF),
    200: Color(0xFFFFFFFF),
    300: Color(0xFFFFFFFF),
    400: Color(0xFFFFFFFF),
    500: Color(0xFFFFFFFF),
    600: Color(0xFFFFFFFF),
    700: Color(0xFFFFFFFF),
    800: Color(0xFFFFFFFF),
    900: Color(0xFFFFFFFF),
  },
);
\end{lstlisting}

%----------------------------------------------------------------------------------------
\section{Widgets}

%----------------------------------------------------------------------------------------
\subsection{Contâineres}
Com \textit{children}:
\begin{itemize}[nolistsep]
	\item {\ttfamily Column(children: [])}
	\item {\ttfamily Row(children: [])}
	\item {\ttfamily Wrap(children: [])} - Quebra a linha
	\item {\ttfamily Stack(children: [\_image, \_text,],)} - Imagem mesclada com texto
\end{itemize}

Com \textit{child}:
\begin{itemize}[nolistsep]
	\item {\ttfamily Container(child: )}
	\item {\ttfamily Card(child: )}
	\item {\ttfamily Align(alignment: Alignment.[tipo], child: )}
	\item {\ttfamily Padding(padding: const EdgeInsets.all(5.0), child: )}
	\item {\ttfamily SafeArea(child: )}
	\item {\ttfamily ClipRRect(borderRadius: BorderRadius.circular(25), child: )}
	\item {\ttfamily Center(child: )} - Centraliza o texto
	\item {\ttfamily FittedBox(child: )} - Maximiza o texto
	\item {\ttfamily Visibility(visible: true/false, child: )} - faz o objeto aparecer ou desaparecer
	\item {\ttfamily Positioned(top: , left: , height: , width: , child: )}
\end{itemize}

Decorar um \textit{Container}:
\begin{lstlisting}[]
 Container(
  decoration: BoxDecoration(
   borderRadius: BorderRadius.circular(8.0),
   color: Colors.blueAccent,
  ),
 ),
\end{lstlisting}

%----------------------------------------------------------------------------------------
\subsection{Divisores}
Obter um determinado espaçamento:
\begin{itemize}[nolistsep]
	\item {\ttfamily Expanded(flex: [fator], child: )}
	\item {\ttfamily Flexible(flex: [fator], child: )}
	\item {\ttfamily SizedBox(height: alt, width: larg, child: )}
	\item {\ttfamily Placeholder(fallbackHeight: 200, color: Colors.blue, strokeWidth: 5)}
\end{itemize}

%----------------------------------------------------------------------------------------
\subsection{Relativos a Imagem}
Imagens podem vir de Asset ou Web: \\
Asset: {\ttfamily Image.asset('assets/images/notfound.png')} \\
Web: {\ttfamily Image.network('endereço', fit: BoxFit.cover)}

Não quebrar uma imagem vinda da Web:
\begin{lstlisting}[]
 Hero(
  tag: 'hero', 
  child: Image.network('https://food.bolt.eu/og-img.jpg'),
 ),
\end{lstlisting}

Criar uma imagem circular:
\begin{lstlisting}[]
 CircleAvatar(
  radius: ,
  backgroundColor: ,
  child: _image,
 )
\end{lstlisting}

Exata proporção de uma imagem:
\begin{lstlisting}[]
 AspectRatio(
  aspectRatio: 1 / 2, (ou 3/4, 7/2)
  child: _image,
 ),
\end{lstlisting}

Mudar a opacidade de uma imagem:
\begin{lstlisting}[]
 Opacity(
  child: _image,
  opacity: 0.3,
 ),	
\end{lstlisting}

Adicionar um filtro de cor:
\begin{lstlisting}[]
 ColorFiltered(
  colorFilter: const ColorFilter.mode(
   Colors.red,
   BlendMode.modulate,
  ),
  child: _image,
 ),	
\end{lstlisting}

Efeito de aparecer (fade) em imagens da Internet:
\begin{lstlisting}[]
 FadeInImage.assetNetwork(
  placeholder: 'images/loading.gif',
  fadeOutDuration: Duration(seconds: 2),
  image: 'https://picsum.photos/250?image=9',
 ),	
\end{lstlisting}

%----------------------------------------------------------------------------------------
\subsection{Relativos a Textos (Labels)}
Texto padrão com estilo:
\begin{lstlisting}[]
 Text(
  episodio['name'] ?? '',
  style: TextStyle(color: corEpisodio, fontSize: 20),
 ),
\end{lstlisting}

Texto selecionável:
\begin{lstlisting}[]
 SelectableText('', style: [estilo])
\end{lstlisting}

Texto com elementos de estilo interno:
\begin{lstlisting}[]
 child: RichText(
  text: TextSpan(
   style: TextStyle(fontSize: 20, color: Colors.black),
   children: <TextSpan>[
    const TextSpan(text: 'Este é um exemplo de '),
    TextSpan(
     text: 'RichText',
     style: TextStyle(
      fontWeight: FontWeight.bold,
      color: Colors.blue[400],
     ),
    ),
   ],
  ),
 ),
\end{lstlisting}

%----------------------------------------------------------------------------------------
\subsection{Relativos as Listas de Widgets}
Barra de rolagem:
\begin{lstlisting}[]
 Scrollbar(
  isAlwaysShown: true,
  child: ...
 ), 
\end{lstlisting}

Slider - Intervalo de Valores: 
\begin{lstlisting}[]
 Slider(
  value: valorCorrente,
  onChanged: (novoValor) {
   setState(() {
    valorCorrente = novoValor; 
   });
  },
  min: 0,
  max: 100,
 ),
\end{lstlisting}

Chip - Lista de algo
\begin{lstlisting}[]
 Chip(
  avatar: CircleAvatar(
   child: Text(emails[index].substring(0, 1)),
  ),
  label: Text(emails[index]),
  onDeleted: () {
   setState(() {
    emails.removeAt(index);
   });
  },
 ),	
\end{lstlisting}

PageView - Várias Visões:
\begin{lstlisting}[]
 PageView(
  scrollDirection: Axis.vertical,
  children: _widgets,
 ),	
\end{lstlisting}

Tabela:
\begin{lstlisting}[]
 DataTable(
  columns: [
   DataColumns(label: nomcol1),
   DataColumns(label: nomcol2),
  ],
  rows: [
   DataRows(cells: [
    DataCell(Text(val1c1)),
    DataCell(Text(val1c2)),
   ]),
   DataRows(cells: [
    DataCell(Text(val2c1)),
    DataCell(Text(val2c2)),
   ]),
  ],
 )
\end{lstlisting}

Lista lateral:
\begin{lstlisting}[]
 return Scaffold(
  ...
  drawer: Drawer(
   child: _listView,
  ),
 );
\end{lstlisting}

%----------------------------------------------------------------------------------------
\subsection{Diversos e úteis}

Dica rápida:
\begin{lstlisting}[]
 Tooltip(
  message: 'Aqui a dica', 
  child: [widget]
 ),
\end{lstlisting}

Future.delayed - Espera um tempo:
\begin{lstlisting}[]
 onPressed: () async {
  await Future.delayed(
   Duration(seconds: 1),
  );
  ...
 }
\end{lstlisting}

Quando tiver uma espera:
\begin{lstlisting}[]
 import 'dart:io'; // import
 import 'package:flutter/cupertino.dart'; // Cupertino

 Platform.isAndroid
  ? CircularProgressIndicator(): CupertinoActivityIndicator()
\end{lstlisting}

Mensagem SnackBar:
\begin{lstlisting}[]
 child: Builder(
  builder: (context) => GestureDetector(
   onTap: () {
    ScaffoldMessenger.of(context).showSnackBar(
     const SnackBar(
      duration: Duration(seconds: 1),
      content: Text('This is the Snackbar'),
     ),
    );
   },
  ),
 ),
\end{lstlisting}

Criar um QrCode:
\begin{lstlisting}[]
 QrImage(
  data: 'o que quiser aqui',
  version: QrVersions.auto,
  size: 200.0,
 ),
\end{lstlisting}


%----------------------------------------------------------------------------------------
\section{Top Pacotes e Plugins}

Página dos Pacotes:
\\
\url{https://pub.dev/}

\url{https://pub.dev/packages/introduction_screen} \\
{\ttfamily\$ flutter pub add introduction\_screen}

\url{https://pub.dev/packages/flutter_native_splash} \\
{\ttfamily\$ flutter pub add flutter\_native\_splash}

\url{https://pub.dev/packages/google_fonts} \\
{\ttfamily\$ flutter pub add google\_fonts}

Ex: 
\begin{lstlisting}[]
 ttfamilyText('Algo', style: GoogleFonts.aguafinaScript().copyWith(fontSize: 40),),
\end{lstlisting}

\url{https://pub.dev/packages/flutter_launch_icons} \\
{\ttfamily\$ flutter pub add flutter\_launch\_icons}

\url{https://pub.dev/packages/fluttertoast} \\
{\ttfamily\$ flutter pub add fluttertoast} 

Ex:
\begin{lstlisting}[]
 onPressed: () {
  Fluttertoast.showToast(
   msg: 'Mensagem da Torrada',
   backgroundColor: Colors.deepOrange,
   textColor: Colors.white,
   fontSize: 16.0,
  );
 },
\end{lstlisting}

\url{https://pub.dev/packages/screenshot} \\
{\ttfamily\$ flutter pub add screenshot}

\subsection{Plugins para o VS Code}
\begin{itemize}[nolistsep]
	\item Dart - Dart Code
	\item Flutter - Dart Code
	\item Material Icon Theme - Phillipp Kief
	\item Pubspec Assist - Jeroen Meijer
	\item Android Emulator Launcher - Dannark
	\item Awesome Flutter Snippets - Neevash Ramdial
	\item Error Lens - Alexander
\end{itemize}

%----------------------------------------------------------------------------------------
\section{Dicas de Dart}

Ao invés de usar: {\ttfamily episodio['name'] == null ? '' : episodio['name']} \\
Converter para: {\ttfamily episodio['name'] ?? ''}

Obter o tamanho da tela:
\begin{lstlisting}[]
 Size size = MediaQuery.of(context).size;
 double height = size.height;
 double width = size.width;
\end{lstlisting}

Ao invés de usar: {\ttfamily String text = '$\setminus$\$100';} \\
Converter para: {\ttfamily String text = r'\$100';}

\end{document}