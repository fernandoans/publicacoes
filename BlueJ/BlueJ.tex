\documentclass[a4paper,11pt]{article}

% Identificação
\newcommand{\pbtitulo}{Editor BlueJ}
\newcommand{\pbversao}{1.0}

\usepackage{../sty/tutorial}

%-----------------------------------------------------------------------------
% INÍCIO DO DOCUMENTO
%-----------------------------------------------------------------------------

\begin{document}
\maketitle % mostrar o título
\thispagestyle{fancy} % habilitar o cabeçalho/rodapé das páginas

%-----------------------------------------------------------------------------
% RESUMO DO ARTIGO
%-----------------------------------------------------------------------------

\begin{abstract}
  % O primeiro caractere deve vir com \initial{}
\initial{B}\textbf{lueJ\cite{bluejoficial} é um ambiente de desenvolvimento integrado para desenvolvimento Java. Foi desenvolvido principalmente para iniciação na programação orientada a objetos. Foi desenvolvido por Michael Kölling com a intenção de facilitar o ensino e o aprendizado da programação orientada a objetos. Projetado e implementado pelas equipes das universidade MonashUniversity, Melbourne, Australia e The University of Southern Denmark, Odense}
\end{abstract}
\vspace{20pt}

%-----------------------------------------------------------------------------
% CONTEÚDO DO ARTIGO
%-----------------------------------------------------------------------------
\section{Parte inicial}
O BlueJ é um ambiente de desenvolvimento criado para auxiliar o ensino da linguagem Java e as técnicas de programação orientada à objetos. Também fornece  ao usuário suporte a interação e experimentação, sendo capaz de ilustrar o comportamento dos objetos em memória durante a execução de um programa. 
\begin{figure}[!htb]
	\centering
	\includegraphics[width=0.3\textwidth]{bluej.png}
	\caption{Logo do BlueJ.}
\end{figure} \\[2mm]
Entre as características básicas do BlueJ estão:
\begin{itemize}
	\item Editor de Texto - possibilita escrever código com um salientador de sintaxe e capacidade para compilar o arquivo; 
	\item Class Browser - possibilita visualizar uma representação gráfica das classes contidas em um projeto;
	\item Console Terminal - possibilita a execução de métodos que escrevem strings na no tela, ou lêem do teclado;
	\item Debugger - possibilita um monitoramento mais elegante do funcionamento de um programa, facilitando a detecção e remoção dos erros "lógicos";
	\item Object Inspector -  possibilita ver interativamente o conteúdo dos atributos dos objetos.
\end{itemize}

\section{Instalação}
O BlueJ é distribuído em três formatos diferentes para os sistemas operacionais: Windows, Mac, Debian e demais sistemas. A instalação para os três sistemas é fácil, completamente automática e auto explicativa. Para o quarto tipo envolve alguns passos a mais, porém para usar o BlueJ é necessário ter o Java Standard Edition - JDK - instalado. Para instalar a JDK ir no site da Oracle\cite{oracleoficial}. \\[2mm]
Para o quarto tipo (obviamente iremos supor que se trata de outra distribuição Linux), então é realizar download do arquivo JAR do site oficial do BlueJ, abrir uma janela do terminal e inserir o seguinte comando: \\
{\ttfamily\$ sudo java -jar bluej-[versao].jar} \\[2mm]
Uma janela de instalação será aberta e devemos trocar o primeiro campo para algo como: "/usr/lib/bluej" (dependerá do seu sistema). Após concluída a instalação, executar o BlueJ com o seguinte comando: \\
{\ttfamily\$ /usr/lib/bluej/bluej}

\section{Criando Classes Java no BlueJ}
Objetos só podem ser criados se existirem classes para serem utilizadas como modelos. A linguagem Java possui uma infinidade de classes para as mais diversas finalidades. No entanto, ao criar um software para resolver um determinado problema, você precisará criar classes específicas do seu software. \\[2mm]
Cada classe precisa obrigatoriamente de um nome. Adicionalmente, ela pode ter características e/ou funcionalidades. Para iniciar, vamos criar uma classe contendo apenas características e ir evoluindo nossas classes ao longo do caminho, à medida que vamos introduzindo conceitos e regras. \\[2mm]
Como primeiro exemplo, considere que o proprietário de uma loja de móveis e eletrodomésticos lhe pede para desenvolver um sistema para controle de vendas. O proprietário precisa inicialmente de um sistema para cadastrar clientes, produtos, funcionários e filiais da loja. \\[2mm]
A partir desta descrição, e sabendo que classes são normalmente representadsa utilizando-se substantivos, podemos identificar que os substantivos em negritos serão algumas das nossas classes. Para iniciarmos, abra o BlueJ e será apresentada a janela principal como na figura abaixo.

\section{Conclusão}
Veja outros artigos que publico sobre tecnologia acessando meu Blog Oficial \cite{fernandoanselmo}.

%-----------------------------------------------------------------------------
% REFERÊNCIAS
%-----------------------------------------------------------------------------

\begin{thebibliography}{2}

\bibitem{bluejoficial} 
  Página do BlueJ \\
  \url{http://bluej.org/}

\bibitem{oracleoficial} 
  Site Oficial do Java Oracle \\
  \url{https://www.oracle.com/technetwork/java/javase/downloads/index.html}

\bibitem{fernandoanselmo} 
  Fernando Anselmo - Blog Oficial de Tecnologia \\
  \url{http://www.fernandoanselmo.blogspot.com.br/}

\end{thebibliography}
  
\end{document}
